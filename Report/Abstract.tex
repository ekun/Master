%!TEX root = main.tex
\section*{Abstract}
\addcontentsline{toc}{chapter}{Abstract}
Agile software development teams work with several different artifacts on a daily basis, and by interacting with these artifacts users are involved in work related experiences. By revisiting these experiences and reflecting upon them, users can evaluate and improve how they solve everyday working tasks. Boud et.al defines reflection as a process where the experience is revisited, feelings are re-attended and the experience is re-evaluated\citep{boudreflection1985}. 
Furthermore work and reflection on work are shown to be strongly connected\citep{Schon1983}\citep{Chaiklin1993}. Reflecting on work experiences give a better understanding of the experience itself, allowing for conclusions and lessons learned to be made. 
Reflection transforms experience into knowledge which can be applied to solve challenges in the everyday working environment. \\*
\\*
The main focus of this thesis was to develop a technological tool to collect project artifacts and connect these to work experiences, in order to enhance reflection both individually and collaboratively in agile software development teams.
The tool was developed using a daily delivery cycle. Design choices were made on the basis of available theory, literature and related tools concerning reflection and agile development. Three evaluations were conducted; A usability study, an expert review with an expert in the field of agile software development and a focus group evaluation consisting of eight software developers working in an agile team. 

The work conducted resulted in a Grails web-application, where users connect their daily experiences with project artifacts collected from a Version-control system. These daily reflection notes can be used both individually and collaboratively in a team as preparation for agile retrospective sessions. The tool continuously collects work-related project artifacts and presents these in order for users to \emph{revisit} their work that day. The application aims to trigger reflection on user experiences and storing the outcome in notes for later use and sharing.

This thesis, the developed tool and its evaluation contributes with an increased understanding of how reflection in agile software development teams can be enhanced, by connecting experiences with work related project artifacts. 

\newpage