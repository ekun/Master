%!TEX root = main.tex
\chapter*{Abstract}
\addcontentsline{toc}{chapter}{Abstract}
Artifacts is a part of our daily routine, we interact and make use of them wherever we go. In a work situation we interact and deal with such artifacts(ie. a report, programming code). By interacting with these artifacts users are involved in learning experiences. By reflecting upon these experiences, users can improve their learning. Boud et.al defines reflection as a process where the experience is revisited, feelings are re-attended and the experience is re-evaluated\cite{boudreflection1985}. Interacting with technological artifacts is one way of supporting this process. 
Work and reflection on work are shown to be strongly connected\cite{Schon1983}\cite{Chaiklin1993}. Reflecting on work experience gives a better understanding of the experience itself, allowing for conclusions and lessons learned to be made. 
Reflection transforms experience into knowledge which can be used to cope with challenges in the everyday working environment. 

In this thesis, focus was on developing a prototype to enhance reflection through the theory of experience based learning, in a collaborative setting. The thesis shows usage of the tool in two different scenarios demonstrating different types of artifacts. Common for both scenarios are that the primary goal was to promote reflection through experience based learning, while aiding traditional learning as a secondary objective. 
The main objective was to offer a tool for reflection around these artifacts in a collaborative environment that enhances learning. 