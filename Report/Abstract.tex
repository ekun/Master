\chapter*{Abstract}
\addcontentsline{toc}{chapter}{Abstract}
Artifacts is a part of our daily routine, interact and work with work related artifacts(ie. a report, programming code). Users can learn more about or improve their artifacts through reflection. Boud et.al defines reflection as a process where the experience is revisited, feelings are re-attended and the experience is re-evaluated\cite{boudreflection1985}. 

Insert reference to artifacts here

In this thesis, focus was on developing a prototype to enhance reflection through the theory of experience based learning, in a collaborative setting. The thesis shows usage of the tool in two different scenarios demonstrating different types of artifacts. Common for both scenarios are that the primary goal was to promote reflection through experience based learning, while aiding traditional learning as a secondary objective. 
The main objective was to offer a tool for reflection around these artifacts in a collaborative environment that enhances learning. 