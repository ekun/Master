\chapter{Introduction}
The last few years personal digital assistants(PDA) or smart-phones have become a gadget we take for granted and need this for our daily activities.  We take notes and make plans during the day, and share this with our friends and family. Our research will take in consideration different methods on how to display these day-to-day notes and how they may affect learning when collaborating on reflection.

For example when a class is on a school-trip and afterwards you want them to write an essay about what they have learned about the destination. If they had the opportunity to take notes they would the ability to go back and read them and use them as a basis for their essay, but let’s say they had the opportunity to use a tool which gives them the ability to collaborate on the notes and being able to review these notes in multiple ways in order to gain the most of them. Being able to see the notes spread out on a map and pinned to the location where they originated will be of great benefit.

Multi-media and multi-representational learning environments are ubiquitous and were so even before the advent of modern educational technology.\cite{Ainsworth1999} By combining three informative ways to display information we give the user an opportunity to view his/her notes or his/her group’s notes in a different way and maybe even put them in perspective. This will help the learners to how to make sense of the information and how to make conscious decision about the use of information. This will give them the opportunity to take a step back and reflect on the information. In order to achieve good results and to further enhance the ability to learn, reflection is needed. These skills are important in both learning with and without technology, but when being supported by technology however, this technology should promote these aspects within learning \cite{Lin1999}.

The goal is to offer technology that not only makes information search and collection efficient, but also support reflection in a collaborative session. Through reflection learning can be achieved and users can collaborate on gathering information for experiences and the ability to reflect over experiences by displaying data in different methods this might enhance the reflection ability and letting the user view data in another way.