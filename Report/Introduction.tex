%!TEX root = main.tex
\chapter{Introduction}
This thesis presents a design-science project where the goal was to design a web-application tool as a support for learning and reflection in agile project development teams. The tool was given the name \emph{PeacefulBanana} and the name has since remained unchanged. The purpose of the tool is to make collection of experiences easier during a project development process, and enhance learning outcomes from reflection sessions. 

The tool collects data from the development process and presents them to the users in a tailored way. Users can annotate these data with their own experiences, including connecting feelings and emotions to these experiences. The application presents these data and experiences for use in both individual and collaborative reflection settings. The challenge will be to collect, annotate and present the data in a way that helps the user understand what they have experienced and learn from these experiences. That is helping them reflect on the experiences made and learn from them. Boud et.al\citep{boudreflection1985} presents that; 
\begin{quote}
\emph{Reflection is a process where the experience is revisited, feelings are re-attended and the experience is re-evaluated}
\end{quote}
This means that it is important to tailor the reflection process in order to support and achieve learning from experiences. Using technology to collect data and experiences from everyday work, and the potential for using this to support reflection has been shown to be growing \citep{li2011understanding}. In this thesis we utilize technology for promoting reflection, by trying to capture everyday data related to work experiences, make them available for reflection and store them for later use. The quality of data collected will have an impact on how easy it is for users to revisit the experience and reflect upon it. By using technology we hope to make the challenge of capturing the experience with it's ideas and feelings to support the reflective process easier to overcome. 

\section{Context \& Domain}
For many years now, agile methods in software development have been widely used in software development teams. Agile methodologies are software development methods that is based on iterative and incremental development. This means that requirements and solutions are dynamic and comes as a result of collaboration between self-organizing teams. Agile methods focuses on adaptive planning, iterative development and encourages teams to respond to changes in a flexible way. 

The agile manifesto presents a principle which states that an agile team should regularly reflect on how to become more effective, and tune its behavior accordingly\citep{Beck2001}. Continuously improving through introspection is a vital part of agile methods and is applied i.e. retrospectives\citep{Beck1999, Derby2006, Maham2008}. Cockburn impose that a vital part of agile practice should be conducting regular retrospective workshops aimed at reflection and process tune-up\citep{Cockburn2006}. Agile methods focus on continuous iterations repeating the same development steps, and thus progressing. Retrospectives in agile development processes are most often performed after each iteration. This is done by gathering the team and reflecting on their way of working, so that improvements for the next iteration can be identified \citep{Derby2006, Drury2011}. This enables agile self-governing teams to react quickly to changes, and make modifications accordingly\citep{Drury2011}. Retrospectives could also support communication and interaction within the team, which is important for agile development. 
 
There are however shortcomings of applying retrospectives in agile teams. Agile teams tend to focus primarily on short-term issues that were identified for a single iteration, and not on long-term strategic issues\citep{Drury2011}. Reflection on work related experiences enables users to reflect and derive conclusions from the reflection \citep{Korthagen_Vasalos_2005}. This turns experiences into knowledge that can be applied to the everyday challenges of work and also creates an individual aspect, where the reflection on previous work can be applied to future work challenges. In addition to this individual aspect, reflection has been shown to have strong social aspects, and is often accomplished collaboratively between users in such agile teams \citep{Høyrup_2004}. Therefore a challenge will be to easily allow users to identify their common tasks and shared work experiences. In software development projects in the industry, it is often a challenge for teams to prioritize retrospectives, as there is other work that is seen as more important \citep{kasi2008post}. For students, revisiting experiences and reflecting on these is often seen as unnecessary and in the way of other tasks, i.e. writing code, testing or documentation. Another challenge is collecting data of sufficient quality to support this reflection, and also how to share this data, experiences and reflection with the rest of the team.  

In this thesis we developed a tool in order to address the challenges of collecting data of sufficient quality to support reflection and how to share these data and experiences in order for users to revisit and reflect upon them. We wish to allow teams to not only reflect on experiences for the last iteration only, but also allow users to go back in time several iterations, in order to identify long-term issues and trends within the team. The tool is aimed at supporting reflection by scaffolding the collection of relevant project artifacts, and prompting users to annotate these artifacts with their experiences and feelings. The tool is designed and developed for software development projects using an agile development process, and will be evaluated with a usability study, a focus group consisting of software developers and also an expert in the field of agile methodologies. 

\subsection{Core Concepts}
In this section we will briefly introduce some concepts which the application developed for this thesis builds upon.
\subsubsection{Learning from experience}
According to David Kolb\citep{KolbModel} for a learning experience to occur there must exists certain abilities in the learner. First the learner must be willing to actively be involved in the experience Secondly the learner must be able to reflect on the experience and third, the learner must possess and be able to use analytical skills to conceptualize the experience. Finally the learner must have skills for decision making and problem solving to be able to create new knowledge outcomes based on the experience.

\subsubsection{Software development}
Software developers generate a lot of data when developing software, when committing data to version control systems or closing tasks and issues in a project management tool. During an agile development cycle, tasks is distributed among the developers and they make decisions either individually or collaboratively. Normally this data is never revisited, but they contain vital information on choices the developers make daily. It is important to make collection of information a part of developers normal day-to-day activities , as can be based on the model from \citep{Krogstie2009}.

\subsubsection{Version-control systems}
Version-control systems is the management of changes to documents, source files or other collections of information. These artifacts are usually identified by a 'revision'\footnote{An unique identifier normally a number or string.}, when creating a revision a lot of data is stored together with the revisioned files. I.e. who committed the data, when it was committed and what files where affected by the change.

In software development, version-control systems can be used for documentation and configuration of a wide variety of files as well as source code. As teams develop software, it is common for developers to create and work on different versions of a system at the same time. Different versions can be tagged in the version-control system, so it is easy to go back in time to a specific state, i.e. looking up the tag for when the last stable version was created.

%!TEX root = main.tex
\section{Research Question}
These are the research questions for our thesis
\begin{itemize}
	\item How to promote experienced-based learning from reflection based on project artifacts collected from version-control systems?
	\begin{itemize}
		\item How to scaffold collection of data in order to promote reflection?
		\item How to bring together contributions from multiple users, and sharing these in a collaborative environment? 
		\item How to increase the tendency to reflect on experiences, both individually and as a team? 
	\end{itemize}
\end{itemize}
%!TEX root = main.tex
%
\section{Research Method}
\subsubsection{Design as an artifact}
By definition, the result of design-science research in Information Science is:
\begin{quote}
\emph{A purposeful IT artifact created to address an important organizational problem. It must be described effectively, enabling its implementation and application in an appropriate domain.} \citep{Esearch2004}
\end{quote}
Markus et al.\citep{markusetal} identified that a developed artifact is only significant if there is questions like: Can it be constructed, can it perform appropriately and is the result important to the information science community. 
We will create a proof of concept prototype tool for promoting experienced-based learning from reflection based on project artifacts collected from version-control systems. Our goal was to create a tool that can discover new capabilities in the domain of reflection and experience-based learning, as well as support the existing capabilities in an efficient way. Evaluating the tool in \emph{real use} situations is necessary in order to discover if the artifact can enhance the reflection process as it is today.

\subsubsection{Research Guidelines}
The \emph{Design-Science Research Guidelines} presented in \citep{Esearch2004} will serve as the basis of this design-science research. The guidelines were created in order to assist researchers to understand the necessary requirements for effective design-science research. 

\subsection{Design as a research process}
Development of the application was done in development cycles, inspired be the regulative cycle presented by Wieringa \citep{wieringa}, and is shown in Figure ~\ref{regulativecycle}. 
The early parts of the development process was used to develop ideas and basic mock-ups of the application and its design. The initial prototype design was based heavily on the first concept and mock-ups created, in addition to recurring feedback from users and supervisors. Early parts of development resulted in a prototype with functionality for individual reflection use. The next goal was to incorporate teams and support collaborative reflection in the team, building and improving on the basic functionality present in the tool. In the later parts of the development, the focus was the integration of individual reflection notes, into the team collaborative data. This resulted in a tool for both individual and collaborative aspects for reflection and learning. We used the tool ourselves during development, as it would be used in a real world environment. In this way we ensure that all functionality is as expected, and also identify limitations in the design. 

Finally we performed a case study evaluation with computer-science students, and an expert review. Evaluating these separately provide two sets of feedback which we can compare in order to possibly see if any patterns emerge. 
\begin{figure}[!htpb]
\centering
	\includegraphics[width=\textwidth]{regulativecycle}
\caption{Regulative cycle development in Design-Science research \citep{wieringa}}
\label{regulativecycle}
\end{figure}

\subsection{Development process}
\begin{figure}[!htpb]
\centering
	\includegraphics[width=\textwidth]{researchmodel_and_iterationcycle}
\caption{The PeacefulBanana research model }
\label{researchmodel}
\end{figure}

Figure \ref{researchmodel} shows the research model used for the development of our prototype from initial concept and ideas, to the final implementation and evaluation. The model depicts how we started the process by reading theory in order to get a better understanding the core concepts of reflection on experiences, and the learning from experiences(See Chapter \ref{chap:background} for the thesis background). We conducted a literature review (Section \ref{sec:literaturereview}) and looked at related work done on computer-supported tools for reflection and learning (Section \ref{sec:relatedwork}). On the basis of theory and previous work done, a problem elaboration was created (Chapter \ref{chap:problemelaboration}). The problem elaboration led us in turn to the process of creating the requirements for the application (Chapter \ref{chap:requirements}). The initial design with concept and mock-ups also helped identify requirements for the application, together with feedback from our supervisor. All implementations were then done conducted using a daily delivery cycle.
\clearpage

\subsection{Daily Delivery Cycle}
\label{sec:dailydeliverycycle}

 \ref{iterationprocess} shows an illustration of our daily delivery cycle. The cycle was inspired by the regulative cycle presented by Wieringa\citep{wieringa}, shown in Figure \ref{regulativecycle}. 
\begin{figure}[!htpb]
\centering
	\includegraphics[width=\textwidth]{iterationprocess}
\caption{The PeacefulBanana daily delivery cycle}
\label{iterationprocess}
\end{figure}

The cycle starts with a set of requirements. These requirements are initially based on user-stories and scenarios. In subsequent iterations of the cycle, the requirements were then updated based on feedback from from experts and users of the application. A task list is then created from the set of requirements, where the highest prioritized requirements was implemented first. This task list acts as a backlog 
\footnote{The backlog is the list of work the developers must address during the current iteration} for the application. On GitHub we created the more general tasks as \emph{Milestones} and the more specific tasks as \emph{issues} connected to these milestones\footnote{GitHub Issues \& Milestones: \url{https://github.com/features/projects/issues}}.\\* Development was conducted primarily using pair-programming\footnote{Pair programming is an agile software development technique in which two programmers work together at one workstation}, so that we could review each line of code as it was written. Whenever a task has been implemented, the implementation was tested manually in the application, as well as by automatic tests. When a feature was accepted and working as intended, we submitted the code to our project repository on GitHub.
This ensured an iterative approach to development, expanding our application with more and more quality-assured functionality. 

\subsubsection*{Dogfooding}
While developing the application, we continuously used the application ourselves, this is a concept called dogfooding.

Dogfooding is a concept where you 'eat your own dog food', this means that the developers are testing their own product while developing it\citep{dogfooding}. It is normally something larger software development teams are doing as a pre alpha-testing, this will help developers to discover eventual problem-areas early.

\subsection{Evaluation} 
Table 2 , shown in Figure \ref{table2evaluation} from \citep{Esearch2004} will serve as guidelines for evaluation of our prototype.
During evaluation we will be using parts of the evaluation toolbox published by MIRROR\footnote{\url{http://www.mirror-project.eu/showroom-a-publications/downloads/finish/5/67}}. This toolbox is a specification of evaluation methodology and research tooling. 

\begin{figure}[H]
\centering
	\includegraphics[width=\textwidth]{table2evaluation}
\caption{Design Evaluation Methods}
\label{table2evaluation}
\end{figure}

\subsubsection{Focus group}
Focus groups is a form of qualitative research which can be compared to semi-structured group interviews\citep{rogers2011interaction}. Participants in a group are asked about their opinions and views towards a product or concept based on their background and experiences\citep{krueger2008focus}. The group should consist of six to ten persons, where participants are not inhibited to present their honest opinions and experiences\citep{krueger2008focus}. The discussion is governed by a facilitator which have the task of keeping focus on the discussion in order to get answers to the questions that have been prepared beforehand\citep{krueger2008focus, nielsen1997use}. The facilitator is responsible for keeping the discussion open, uninhibited, non-judgemental and also making sure that all participants are allowed to present their views\citep{powell1996focus}. \\
A focus group usually lasts between 90 and 120 minutes, and is conducted in a neutral place. Limitations of a focus group include that you only get information on what potential users say, and not what they do or if it aligns with the reality\citep{nielsen1997use}. Another limitation is that users often think they need something else than what they really need. Therefore it is important to demonstrate something concrete, where users understand fully what the application is, what it does and what it achieves. 
 
\subsubsection{Observational}
Before and after the user evaluation of the tool, all test-subjects will be given a form with a set of questions to evaluate the application. We will be using core demographic questions, app usage, the most relevant app-specific questions, and also the reflection scale from the toolbox[Appendix \ref{reflectionscale}]. The reflection scale assesses participants' general tendency to reflect and the importance they place on reflection. Using this scale, allows us to see whether using the tool prime people to reflect more. That is: Does the tool increase the users tendency to reflect on experiences, both individually and as a team. Measuring this on users before and after implementation of the tool, allows us to measure the expected increase. \\*
After the evaluation period,  we will include the learning outcomes and the work behavior questions from the evaluation toolbox.

\subsubsection{Analytical}
We will perform a dynamic analysis on the data collected from the application-server and user feedback. During evaluation the application will provide simple usage-patterns on how the users use the application. The usage statistics collected by the application are: 
\begin{itemize}
\item Total number of reflection notes
\item Number of reflection notes per team
\item Number of reflection notes per user
\item The average mood, on a scale from 1 - 100.
\item The ratio of shared notes to private notes. 
\end{itemize}
An example of this data can be seen in Figure: \ref{notestatistics}. 
\begin{figure}[!htpb]
\centering
	\includegraphics[width=\textwidth]{notestatistics}
\caption{Reflection note statistics}
\label{notestatistics}
\end{figure}
\\*
We will also gather statistical data regarding the GitHub data present in the application, shown in Figure: \ref{githubstatistics}
\begin{figure}[!htpb]
\centering
	\includegraphics[width=\textwidth]{githubstatistics}
\caption{GitHub statistics}
\label{githubstatistics}
\end{figure}

This will also help us in the analysis of the applications performance as well as its usefulness in the reflection and learning domains.

\subsubsection{Testing}
During development we will continuously test the tool to ensure that the user experience is as expected, as a means of quality assurance.
White Box testing will be executed during development of the application with artificial data to ensure that the function is working properly and as specified. 

\subsubsection{Descriptive}
Evaluation proposal with the students in IT2901 - A project course taken by computer science students on the Norwegian University of Science and Technology (NTNU). 
Students that accept to evaluate our tool, will have the possibility to use it for two weeks, with daily individual use and as a team after the two week period. This will allow us to measure the usage of our scenarios. 

Requirements: 
The first scenario does not require anything but a GitHub repository and is doable both for single and for multiples in teams.
For scenario 2 we will require a team of at least 3 people in order to collect enough data. 
We have continuously been using and evaluating the tool ourselves, also a fellow student has been using it for a couple of weeks, so we have some additional evaluation
data also.

\subsection{Research Contributions}
The main goal was to create a proof of concept application that may help users reflect on their past experiences, both individually and collaboratively in reflection sessions, by collecting experiences daily, storing and revisiting them at any time. The implementation of the application and the evaluation of it gave us a foundation to answer our research questions, and also identify areas that can go into future research and development in the domain of technology supported reflection and collaboration. 
Future work  and the identified proposed features and improvements can be seen in Section \ref{sec:futurework}

 will provide an increased understanding of how technology can support reflection in software development teams, by collecting and revisiting experiences from everyday work. 

\subsection{Research Rigor}
The application was evaluated through a usability test, an expert review and a focus group consisting of a software development team. By evaluating the application with experts in the relevant fields provides an indication for the usefulness of the application.

\subsection{Research communication}
The proof of concept application have been released under the GNU Public License v3 and the repository are located at Github: 
\begin{itemize}
	\item \url{https://github.com/ekun/PeacefulBanana}
\end{itemize} 
Any limitations identified  will be documented along with the research results. 




\section{Outline}
In this section we will describe the organization of the remaining chapters in our thesis. 
\\*
\\*
\textbf{Chapter 2} gives a general overview of the PeacefulBanana application. The intention is to present features and concepts of the application on an abstract level.
\\*
\\*
\textbf{Chapter 3} describes the theoretical background behind over thesis, including reflection, experiences and experience based learning. 
\\*
\\*
\textbf{Chapter 4} describes our State-of-the-art, the literature review that was performed and related work. The chapter presents some of the work performed in the domains of technology for reflection and experience based learning, with related work acting as a basis for our own implementation of technology. 
\\*
\\*
\textbf{Chapter 5} presents the problem definition, an introduction to GitHub, user stories and our scenarios. 
\\*
\\*
\textbf{Chapter 6} describes the requirements chapter, elaborated from problem elaboration and the scenarios. 
\\*
\\*
\textbf{Chapter 7} describes the design choices behind the creation of the PeacefulBanana tool.
\\*
\\*
\textbf{Chapter 8} describes how we implemented the PeacefulBanana tool. 
\\*
\\*
\textbf{Chapter 9} includes a description of the different evaluations of the application, the usability study, the expert review and the focus group. The chapter also presents a discussion of the results gathered and how they may answer our research questions.
\\*
\\*
The last chapter of our thesis, is a conclusion with a summary, an evaluation of our own work and ideas towards future work.