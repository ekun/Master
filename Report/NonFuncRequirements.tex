%!TEX root = main.tex
%• Functional Requirements: Detail the requirements that you have to fulfill in order
%to complete the task.
\chapter{Non Functional Requirements}
\label{cha:nonfuncreq}
Non functional requirements of the application

\section{Usability}
According to Bass, Clements and Kazman, \emph{usability} is concerned about making the the tasks involved in the system as easy to accomplish as possible \cite[p.~90]{pensum}. Below we have identified the following use cases for assuring the usability of the application.

\begin{itemize}
    \item[\textbf{U1}] Support multiple devices \\
    \textit{\small{The users should be able to access the application on any device and screen size.}}
        
    \begin{tabular}{| l | p{7cm} |}
        \hline
        \rowcolor[gray]{0.8}
        \textbf{Portion of scenario} & \textbf{Values} \\
        \hline
        Source & End User \\
        Stimulus & Change position of the ships \\
        Artifact & System \\
        Environment & Game play \\
        Response & Move ships  \\
        Response measure    & After positioning the ships, the user is able to change these positions, one  at a time. \\
        \hline
    \end{tabular}

    \item[\textbf{U2}] Something \\
    \textit{\small{The user shall be able to see each of their ships being placed on the ocean.}}
        
    \begin{tabular}{| l | p{7cm} |}
        \hline
        \rowcolor[gray]{0.8}
        \textbf{Portion of scenario} & \textbf{Values} \\
        \hline
        Source & End User \\
        Stimulus & Change position of the ships \\
        Artifact & System \\
        Environment & Game play \\
        Response & Move ships  \\
        Response measure & After positioning the ships, the user is able to change these positions, one  at a time. \\
        \hline
    \end{tabular}

    \item[\textbf{U3}] The application shall show hints whereever appropriate. \\
        \textit{\small{The user gets well-founded recommendations, tips or warnings during use, in order to increase confidence.}}
        
        \begin{tabular}{| l | p{7cm} |}
            \hline
            \rowcolor[gray]{0.8}
            \textbf{Portion of scenario} & \textbf{Values} \\
            \hline
            Source & End User \\
            Stimulus & User is uncertain on how the application is used, or what their next move can be \\
            Artifact & System \\
            Environment & Run-time \\
            Response & A message box with tips \\
            Response measure  & User can use the application without problem regarding to applitions usebility in \emph{1 hour}\\
            \hline
        \end{tabular}
\end{itemize}

\section{Availability}
According to someone availability is concerned about making the application as availible as possible

\begin{itemize}
    \item[\textbf{A1}] Uptime \\
    \textit{\small{The application shall be availible more than 90\% of the time.}}
        
    \begin{tabular}{| l | p{7cm} |}
        \hline
        \rowcolor[gray]{0.8}
        \textbf{Portion of scenario} & \textbf{Values} \\
        \hline
        Source & End User \\
        Stimulus & Accessing website \\
        Artifact & System \\
        Environment & Website \\
        Response & Show the user the website it desires. \\
        Response measure & Webserver responds with the application when the server is running. \\
        \hline
    \end{tabular}

    \item[\textbf{A2}] Persistent storrage \\
    \textit{\small{When the system is rebooted it needs to restore to the same state as before.}}
        
    \begin{tabular}{| l | p{7cm} |}
        \hline
        \rowcolor[gray]{0.8}
        \textbf{Portion of scenario} & \textbf{Values} \\
        \hline
        Source & End User \\
        Stimulus & Accessing website \\
        Artifact & System \\
        Environment & Website \\
        Response & Show the user the website it desires. \\
        Response measure & Webserver responds with the application when the server is running. \\
        \hline
    \end{tabular}
\end{itemize}

\section{Security}


\begin{itemize}
    \item[\textbf{S1}] Secure data storage \\
    \textit{\small{The application shall store sensitiv data secured.}}
        
    \begin{tabular}{| l | p{7cm} |}
        \hline
        \rowcolor[gray]{0.8}
        \textbf{Portion of scenario} & \textbf{Values} \\
        \hline
        Source & Developers \\
        Stimulus & Add security messures. \\
        Artifact & System \\
        Environment & Run time \\
        Response & Secure data  \\
        Response measure & After positioning the ships, the user is able to change these positions, one at a time. \\
        \hline
    \end{tabular}

    \item[\textbf{S2}] Authentication \\
    \textit{\small{The users shall be authenticated to view sensitiv data.}}
        
    \begin{tabular}{| l | p{7cm} |}
        \hline
        \rowcolor[gray]{0.8}
        \textbf{Portion of scenario} & \textbf{Values} \\
        \hline
        Source & End User \\
        Stimulus & Requesting to view sensitiv data. \\
        Artifact & System \\
        Environment & Run time \\
        Response & Check authentication \\
        Response measure & If authenticated show the appropriate data. \\
        \hline
    \end{tabular}

\end{itemize}