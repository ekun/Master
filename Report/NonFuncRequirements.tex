%!TEX root = main.tex
%• Functional Requirements: Detail the requirements that you have to fulfill in order
%to complete the task.
\section{Non Functional Requirements}
\label{cha:nonfuncreq}
Non functional requirements of the application.

\subsection{Usability}
According to Bass, Clements and Kazman, \emph{usability} is concerned about making the the tasks involved in the system as easy to accomplish as possible \citep[p.~90]{ProgarkPensum}. Below we have identified the following use cases for assuring the usability of the application.

\begin{itemize}
    \item[\textbf{U1}] Support multiple devices \\
    \textit{\small{The users should be able to access the application on any device and screen size.}}
        
    \begin{tabular}{| l | p{7cm} |}
        \hline
        \rowcolor[gray]{0.8}
        \textbf{} & \textbf{Values} \\
        \hline
        Source & Developer \\
        Stimulus & Make design responsive \\
        Artifact & System \\
        Environment & Run-time \\
        Response & Scale after resolution  \\
        Response measure & The application shall fit to the screen size \\
        \hline
    \end{tabular}   

    \item[\textbf{U2}] Intuitive design \\
    \textit{\small{The user interface must be easy to understand and not to create any confusion.}}
        
    \begin{tabular}{| l | p{7cm} |}
        \hline
        \rowcolor[gray]{0.8}
        \textbf{} & \textbf{Values} \\
        \hline
        Source & Developer \\
        Stimulus & Make user interface easy to use \\
        Artifact & System \\
        Environment & User interface \\
        Response & Place buttons in a 'natural' matter. \\
        Response measure & Users will find the link they are looking for more than 95\% of the time. \\
        \hline
    \end{tabular}

    \item[\textbf{U3}] The application shall show hints where ever appropriate. \\
        \textit{\small{The user gets well-founded recommendations, tips or warnings during use, in order to increase confidence.}}
        
        \begin{tabular}{| l | p{7cm} |}
            \hline
            \rowcolor[gray]{0.8}
            \textbf{} & \textbf{Values} \\
            \hline
            Source & End User \\
            Stimulus & User is uncertain on how the application is used, or what their next move can be \\
            Artifact & System \\
            Environment & Run-time \\
            Response & A message box with tips \\
            Response measure  & User can use the application without problem regarding to the applications usability in \emph{1 hour}\\
            \hline
        \end{tabular}
\end{itemize}

\subsection{Availability}
Is concerned with the system failure and its associated consequences. A system failure occurs when the system no longer delivers a service consistent with its specification\citep{ProgarkPensum}. Below we have identified the following use cases for assuring the availability of the application.

\begin{itemize}
    \item[\textbf{A1}] Uptime \\
    \textit{\small{The application shall be available more than 90\% of the time.}}
        
    \begin{tabular}{| l | p{7cm} |}
        \hline
        \rowcolor[gray]{0.8}
        \textbf{} & \textbf{Values} \\
        \hline
        Source & End User \\
        Stimulus & Accessing website \\
        Artifact & System \\
        Environment & Website \\
        Response & Show the user the website it desires. \\
        Response measure & Web server will be working correctly more than 90\% of the time. \\
        \hline
    \end{tabular}

    \item[\textbf{A2}] Persistent storage \\
    \textit{\small{When the system is rebooted it needs to restore to the same state as before.}}
        
    \begin{tabular}{| l | p{7cm} |}
        \hline
        \rowcolor[gray]{0.8}
        \textbf{} & \textbf{Values} \\
        \hline
        Source & End User \\
        Stimulus & Accessing website \\
        Artifact & System \\
        Environment & Website \\
        Response & Show the user the website it desires. \\
        Response measure & After shutdown the application shall recover to the same state as before every time. \\
        \hline
    \end{tabular}
\end{itemize}

\subsection{Security}
According to Bass, Clements and Kazman, \emph{security} is a measure of the systems ability to resist the unauthorized usage while still providing its services to legitimate users\citep{ProgarkPensum}. Below we have identified the following use cases for assuring the security of the application.

\begin{itemize}
    \item[\textbf{S1}] Secure data storage \\
    \textit{\small{The application shall store sensitive data secured.}}
        
    \begin{tabular}{| l | p{7cm} |}
        \hline
        \rowcolor[gray]{0.8}
        \textbf{} & \textbf{Values} \\
        \hline
        Source & Developers \\
        Stimulus & Add security measures. \\
        Artifact & System \\
        Environment & Run time \\
        Response & Secure data  \\
        Response measure & After positioning the ships, the user is able to change these positions, one at a time. \\
        \hline
    \end{tabular}

    \item[\textbf{S2}] Authentication \\
    \textit{\small{The users shall be authenticated to view sensitive data.}}
        
    \begin{tabular}{| l | p{7cm} |}
        \hline
        \rowcolor[gray]{0.8}
        \textbf{} & \textbf{Values} \\
        \hline
        Source & End User \\
        Stimulus & Requesting to view sensitive data. \\
        Artifact & System \\
        Environment & Run time \\
        Response & Check authentication \\
        Response measure & If authenticated show the appropriate data. \\
        \hline
    \end{tabular}
\end{itemize}