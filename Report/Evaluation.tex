%!TEX root = main.tex
\chapter{Evaluation}
In this chapter we will describe how we evaluated the PeacefulBanana tool, and how it fulfills the requirements.

% This chapter evaluates your prototype and how it fulfills the
% stated requirements. Depending on the focus and on available resources
% the evaluation might involve users. If you do not have resources for a user
% evaluation, you can have a scenario that helps illustrate how the prototype
% fulfills the requirements and discuss limitations.


\section{Usability}
% Her bør vel det nevnes det fra testplanen, kjapt hva vi ville og resultatene stå. Vi bør også forklare testplanen mer et annet sted, kanskje i research method? Bør vel ikke stå altfor mye i intro kapitlet dog. 
In this section we will describe the results and observations gained through usability tests. The usability test-plan can be seen in : %link to appendix here%

\subsection{Test Goals}
Performing usability tests on the PeacefulBanana prototype serves primarily two goals:
\begin{itemize}
\item Identify possible problems or breakdowns in the design\cite{ref:30} early on in the design process.
\item Identify the relationship users have with the aspect of reflection and sharing personal experiences.
\end{itemize}
Typical problems identified would be text representations or the placement of design elements, that are not intuitive for the user during use. It would be a concern if the user can't figure out how to use certain features of the application. Identifying these problems as early as possible will lead to a better end design. \\
Secondly an objective of the usability test was to identify how users act and think about their daily experiences, how they react to the notion of reflecting on them and if sharing their private thoughts is a problem. 
%\section{Expert Review}
%Evaluate results from expert review. % Birgit and who else?

\section{Interview}
Interview a set of 'real' developers.

\section{Case Study}
The prototype will be evaluated with a group of NTNU students, in the IT2901 bachelor - project. They will use it for a minimum of two weeks, in order to accomodate for both scenarios.

The second scenario is only available for testing by teams of at least 3 people.

Evaluation:
Core questions described in the toolbox, 9.1:
Demographic questions, usage of the app, most relevant app-specific questions - choosing the questions that are most important to our thesis. 
In addition we will use the reflection scale, shown in 9.1.4 both before and after the evaluation period, in order to measure the effect on the tendency to reflect on past
experiences. Finally we will include the learning outcomes questions and the work behaviour questions, after the evaluation period. 

Evaluation proposal with the students in IT2901: 
Students that accept to evaluate our tool, will use it for two weeks, with daily individual use and as a team after the two week period. This will allow us to measure the
usage of both scenario 1 and scenario 2.

Requirements:
The first scenario does not require anything but a github repository and is doable both for single and for multiples in teams.

For scenario 2 we will require a team of at least 3 people in order to collect enough data. 

We have continuously been using and evaluating the tool ourselves, also Trond-Kjetil has been using it for a couple of weeks, so we have some additional evaluation
data also. 

%% Flett Testing seksjonen inn under f.eks usability etc

\section{Testing}
\subsubsection{Dogfooding}
While developing the applications all of the developers used the application, this is a concept called dogfooding.

Dogfooding is a concept where you 'eat your own dog food', this means that the developers are testing their own produkt while developing it. It is normaly something larger software development teams are doing as a pre alpha-testing, this will help developers to discover eventual problem-areas early.

\subsubsection{User interface}
When most of the functionality was finished an expert review of the user interface was conducted.

Explain how we conducted the test.

During the first set of testing it was revealed a set of inconsistencies, mainly regarding where the users could find certain data and where they though it was natural to find them. 
This resulted in a redesign of certain aspects of the application, where data representation was moved after the second series of test subjects confirmed that they where placed unnatural.