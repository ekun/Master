%!TEX root = main.tex
\chapter{Evaluation}
\label{cha:evaluation}
In this chapter we will describe how we evaluated the PeacefulBanana tool, and how it fulfills the requirements identified in Chapter \ref{chap:requirements}.

%!TEX root = main.tex
% Legger alt av usability stuff her enn så lenge

% Bruk testplan.docx og test-report.docx som en slags mal til hva som skal være med i plan og rapport. Legg det inn her 
% Se på Usability Master 2 Svanaes.pdf for et forslag til oppsett av dette. F.eks en section med gjennomføring og en med resultater. Slå sammen stuffet fra testplan.docx og test-report.docx til en fin ting!  
\chapter{Usability}
% Her bør vel det nevnes det fra testplanen, kjapt hva vi ville og resultatene stå. Vi bør også forklare testplanen mer et annet sted, såvidt i research method? Men kanskje også reintrodusere test-kapitlet og legge alt fra intro til resultater der? Bør vel ikke stå altfor mye i intro kapitlet dog
% Usability testen er jo ikke den primære evalueringen, men iom. at vi har begrenset med data bør vi nok fokusere litt på den. 
In this section we will describe the results and observations gained through usability tests. Before we conducted the usability test we created a usability test plan: Appendix \ref{chap:usability}, where all the different parts of the usability test is described in detail.
\section{Method}
The usability test was conducted on four participants. User-interaction with the PeacefulBanana was done through an Internet Browser. Users answered an entrance questionnaire, in order to collect demographic information. During the usability test we took notes of the user's problems and concerns. When the test was completed, participants could comment with suggestions on improvement of the application or design. Finally we had participants answer a SUS form, which consisted of 10 questions designed to measure user satisfaction. 

\subsection{Context}
\label{subsec:context}
The usability test simulated the two scenarios identified in Section \ref{sec:scenarios} and was set in the context of these. We conducted the usability test in order to answer several important questions, regarding these scenarios: 
\begin{itemize}
	\item Is the application easy to use, and can users achieve their goals in a timely manner?
	\item Identify the relationship users have with the aspect of reflection and sharing personal experiences.
	\item Does the tool present data in a way that triggers reflection for the user?
\end{itemize}
Feedback from the usability test was used to further aid design and help identify problem areas that might cause problems for potential users. Since the participants were all computer science experts, and familiar with reflection we hoped to collect valuable input regarding these concepts.

\subsection{Participants}
As mentioned we had four participants in the usability test. As the PeacefulBanana tool is intended to be used with developers with a computer-science background, participants were master students on the Computer Science field at NTNU. These participants all had a background from Computer Science, and were familiar with usability testing and also experienced with the notion of reflection from earlier projects using agile methodologies.
The participants' responsibilities were to attempt to complete a set of representative task scenarios presented to them in as efficient and timely a manner as possible, and to provide feedback regarding the usability and acceptability of the user interface. The participants was directed to provide honest opinions regarding the usability of the application, and to participate in post-session subjective questionnaires and debriefing.

\subsection{Procedure}
The usability test took place in a private room at the university. A computer with the PeacefulBanana web application was used in a typical working environment. The participant’s interaction with the application was monitored by the facilitator seated in the same room.
The facilitator briefed the participants on the web application and instructed the participants that are evaluating the application, rather than evaluating the participant. Participants signed an informed consent that acknowledges: \emph{the participation is voluntary, that participation can cease at any time, and that their privacy of identification will be kept safe}. \\
The facilitator explained that the amount of time taken to complete the test task will be measured and that exploratory behavior outside the task flow should not occur until after task completion. At the start of each task, the participant read aloud the task description from the printed copy and began the task. Time-on-task measurement began when the participant started the task.
The facilitator instructed the participant to ‘think aloud’ so that the facilitator could observe and take notes of user behavior and user comments.
After all task scenarios are attempted, the participant completed the post-test satisfaction questionnaire.

\subsection{Roles}
For our usability test we had two roles, in addition to the test participants: \\
\subsubsection{Facilitator}
	\begin{itemize}
		\item Provides overview of study to participants.
		\item Defines usability and purpose of usability testing.
		\item Responds to participant's requests for assistance.
	\end{itemize}
\subsubsection{Test Observer}
	\begin{itemize}
		\item Silent observer
		\item Takes notes of identified problems, concerns, coding bugs, and procedural errors.
		\item Serve as note takers.
	\end{itemize}

\subsection{Ethics}
All persons involved with the usability test were required to adhere to the following ethical guidelines:
\begin{itemize}
	\item The performance of any test participant must not be individually attributable. Individual participant's name should not be used in reference outside the testing session.
	\item A description of the participant's performance should not be reported. 
\end{itemize}

\subsection{Usability Tasks}
The usability tasks were derived from our scenarios, described in section \ref{sec:scenarios}. Due to the short time for which each participant was available, the tasks used were the most common and relatively complex of available functions. The tasks were identical for all participants in the study.
The application was tested in a development environment and databases were populated during use, and not pre-populated. This ensured a similar experience as to what the users would get when they first use PeacefulBanana in a real-life setting. The web application was run on a local computer, and not on a dedicated server as it was when deployed in production. This and the possible extra overhead from development mode, may have an impact on performance slightly in a negative way.

\subsubsection{Task context}
PeacefulBanana is a tool intended to promote reflection and allow for revisiting and learning from previous experiences. PeacefulBanana integrates with and collects data from the version-control system GitHub.
\subsubsection{Scenario 1 tasks:}
Here are the tasks participants were to solve related to Scenario 1:
\begin{itemize}
	\item Task 1: You start the application for the first time, and want to login, link your account with Github and set an active repository.
	\item Task 2: 
		\begin{itemize}
			\item Task 2.1: View your notifications.
			\item Task 2.2: Find the \emph{“Congratulations”} notification and archive it. Find the archive and see if the notification was indeed archived
		\end{itemize}
	\item Task 3:
		\begin{itemize}
			\item Task 3.1: Find the \emph{“Reminder: Daily Reflection”} note and perform the daily summary.
			\item Task 3.2: Find a daily summary note and share it. Verify that is has indeed been shared.
			\item Task 3.3: Find your mood graph
		\end{itemize}
\end{itemize}

\subsubsection{Scenario 2 tasks:}
Here are the tasks participants were to solve related to Scenario 2:
\begin{itemize}
\item Task 4:
	\begin{itemize}
		\item Task 4.1: Create a team with the name \emph{“Tuttifrutti”} and your previously chosen repository.
		\item Task 4.2: Find your created team and set it to active.
		\item Task 4.3: Identify the members on your team and their role.
	\end{itemize}
\item Task 5:
	\begin{itemize}
		\item Task 5.1: Find all your repositories’ milestones.
		\item Task 5.2: Identify your overdue milestones.
		\item Task 5.3: Find your repositories issues.
		\item Task 5.4: Find issue \#17 . Find the status of this issue, when it was opened and when it was closed.
	\end{itemize}
\item Task 6:
	\begin{itemize}
		\item Task 6.1: Generate a tagcloud for your current chosen repository.
		\item Task 6.2: Identify the most used tag for your team and yourself.
		\item Task 6.3: Find the commit impact for your repository.
	\end{itemize}
\end{itemize}

\subsection{Usability Metrics}
Usability metrics refers to user performance measured against specific performance goals necessary to satisfy usability requirements. Scenario completion success rates, error rates, and subjective evaluations was collected. Time-to-completion/Time-on-task was also collected.

\subsubsection{Task Completion}
Each task requires that the participant obtains or inputs specific data that would be used in course of a typical task. The task is noted as \emph{completed} when the participant indicates the task's goal has been obtained (whether successfully or unsuccessfully).  If a participant requires assistance in order to achieve a correct output then the task will be noted as a critical error and the overall completion rate for the task will be affected.

\subsubsection{Critical Errors}
A critical error is an error that results in an incorrect or incomplete outcome.. Obtaining or otherwise reporting of the wrong data value due to participant workflow is a critical error. Participants may or may not be aware that the task goal is incorrect or incomplete. In general, critical errors are unresolved errors during the process of completing the task or errors that produce an incorrect outcome.

\subsubsection{Non-Critical Errors}
Non-critical errors are errors that are recovered from by the participant or, if not detected, do not result in processing problems or unexpected results. Although non-critical errors can be undetected by the participant, when they are detected they are generally frustrating to the participant.

\subsubsection{Subjective Evaluations}
Subjective evaluations regarding ease of use and satisfaction was collected via questionnaires, and during debriefing at the conclusion of the session. The questionnaires utilized free-form responses and rating scales.

\subsubsection{Task Completion Time(time on task)}
The time to complete a task is referred to as "time on task", not including subjective evaluation duration. It is measured from the time the person begins the task to the time the participant indicates completion.

\subsection{Usability Goals}
The next section describes the usability goals for \emph{PeacefulBanana} in context of the usability metrics. First the general usability test objectives are described, and what usability metric results we aimed for.\\ 
The goals of usability testing the PeacefulBanana application included establishing a baseline of user performance, validating user performance measures, and identifying potential design issues that needed to be addressed in order to improve efficiency, usability and end-user satisfaction. \\
The general usability test objectives were:
\begin{itemize}
	\item Identify possible problems or breakdowns in the design\cite{ref:30} early on in the design process. Sources of such breakdowns may include:
		\begin{itemize}
			\item Navigation errors – failure to locate functions, excessive actions to complete a function, failure to follow recommended screen flow.
			\item Presentation errors – failure to locate and properly act upon desired information in screens, selection errors due to labeling ambiguities.
			\item Control usage problems – improper toolbar or entry field usage.
		\end{itemize}
	\item Exercise the PeacefulBanana application under controlled test conditions with representative users, which here are individuals with a background in IT. Data will be used to assess whether usability goals regarding an effective, efficient, and well-received user interface have been achieved.
	\item Establish baseline user performance and user-satisfaction levels of the user interface for future usability evaluations.
\end{itemize}
The PeacefulBanana application has been developed with developers in mind, and will be evaluated on students in the field of Computer Science, developing in an agile team. The testing will occur in a controlled environment in a private room.

Typical problems identified in a usability test would be text representations or the placement of design elements, that are not intuitive for the user during use. It would be a concern if the user can't figure out how to use certain features of the application. Identifying these problems as early as possible will lead to a better end design. \\
Secondly an objective of the usability test was to identify how users act and think about their daily experiences, how they react to the notion of reflecting on them and if sharing their private thoughts is a problem. 

\subsubsection{Completion Rate}
A completion rate of 100\% is the goal for each task in this usability test. \\
Completion rate is the percentage of test participants who successfully complete the task without critical errors, an \emph{output} that is correct. If a participant requires assistance in order to achieve a correct output then the task will be scored as a critical error and the overall completion rate for the task will be affected.

\subsubsection{Error-Free rate}
An error-free rate of 75\% is the goal for each task in this usability test. \\
Error-free rate is the percentage of test participants who complete the task without any errors (critical or non-critical errors). A non-critical error is an error that is not critical to get correct task output but would result in the task being completed less efficiently.

\subsubsection{Subjective Measures}
Opinions of participators regarding specific tasks, time to perform each task, features, and functionality was collected. At the end of the usability test, participants rated their satisfaction with the overall system. Combined with the interview/debriefing session. 

\subsection{Problem Severity}
In order to analyze collected data from the usability test, identified issues were classified by issue severity. This issue severity is a combination of the impact of the issue and the frequency of users experiencing the issue during the test. 

\subsubsection{Impact}
\subsubsection{Frequency}
	\begin{itemize}
		\item High: 40\% or more of the participants experienced the problem.
		\item Moderate: 20\% - 39\% of participants experienced the problem.
		\item Low: 20\% or fewer of the participants experienced the problem
	\end{itemize}
\subsubsection{Problem Severity Classification}
	\begin{itemize}
		\item High severity: High impact problems that often prevent a user from correctly completing a task. Reward for resolution is reduced redevelopment costs.
		\item Medium severity: Either moderate problems with low frequency or low problems with moderate frequency; these are minor annoyance problems faced by a number of participants. Reward for resolution is typically exhibited in reduced time on task and increased data
		\item Low severity: Low impact problems faced by few participants; there is low risk to not resolving these problems. Reward for resolution is typically exhibited in increased user satisfaction.
	\end{itemize}

\section{Usability Test Results}
\subsection{Pilot Test}
After finalizing the usability test plan: Appendix \ref{chap:usability}, a pilot test was conducted prior to the usability-test[ref 22]. The pilot test allows for an evaluation of the test plan itself and the questionaires before doing the actual usability test. This means the pilot test is a "test of the test", where the goal is to evaluate and verify that the test itself is well-formulated. We chose a fellow student as our pilot-tester, in order to check whether the test script was clear, that the tasks were appropriately difficult, and that the data collected can be meaningfully analyzed. 
It also allows the "tester" to practice the execution and guidance, before actually performing the tests. \\
In the pilot test for PeacefulBanana, all of the aspects above were evaluated and a few tweaks were made to the tests, making it more streamlined. Also a few, smaller bugs in the application were discovered and fixed. The test introduction was rewritten, since the pilot-tester showed some confusion in a few of the tasks. 
The findings acted as valuable feedback to our delivery cycle, and were used for improving the design of the application. 

\section{Summary}
We conducted an onsite usability test using a production version of PeacefulBanana, located on the test administrator’s local server. One tester took notes of comments, facial expressions and navigation choices.The administrator acted as guidance during the test. The sessions captured each participant’s navigational choices, task completion rates, comments, overall satisfaction ratings, questions and feedback.
The usability test was conducted in a private lab-room at NTNU on November 10th. The purpose of the test was to assess the usability of the web application design, information flow, information architecture and the effects of sharing personal reflection notes.
Four participants attended the test. Typically, three to five test participants is the optimal number for most usability studies\cite{nielsen1993mathematical}. Each individual session lasted approximately twenty minutes.
This section contains the participant feedback, satisfactions ratings, task completion rates, ease or difficulty of completion ratings, time on task, errors, and recommendations for improvements.
\subsection{Presentation of results}
\subsubsection{Task Completion Success Rate}
\textbf{Scenario 1 completion rates:}
\begin{figure}[h!]
    \centering
        \includegraphics[width=\textwidth]{scenario1completionrate}
    \caption{Scenario 1 Completion Rate}
    \label{scenario1completionrate}
\end{figure}
\begin{figure}[h!]
    \centering
        \includegraphics[width=\textwidth]{scenario2completionrate}
    \caption{Scenario 2 Completion Rate}
    \label{scenario2completionrate}
\end{figure}
All participants successfully completed (100\% Completion rate):
\begin{itemize}
	\item Task 1 - Start the application and set active repository. 
	\item Task 2.1 - View notifications. 
	\item Task 3.1 - Find the \emph{Reminder: Daily reflection} note and perform the daily summary. 
	\item Task 3.2 - Find and share the daily reflection note.
\end{itemize}
For Task 2.2(Find and archive congratulations notification) and 3.3(Find mood graph), 3 out of 4 participants completed the tasks(75\% Completion rate). \\

\textbf{Scenario 2 completion rates:}
All participants successfully completed (100\% Completion rate):
\begin{itemize}
	\item Task 4.1 - Create a team. 
	\item Task 4.2 - Set active team. 
	\item Task 4.3 - Identify team members. 
	\item Task 5.2 - Identify overdue milestones. 
	\item Task 5.4 - Find issue \#17
	\item Task 6.1 - Generate tagcloud
	\item Task 6.2 - Identify most used personal tags and team tags
	\item Task 6.3 - Find commit impact
\end{itemize}
3 out of 4 participants(75\%) successfully completed Task 5.1(Find all milestones for your repository), while 2 out 4(50\%) successfully completed Task 5.3 (Find your repositories issues). 

\subsubsection{Time on task}
Time on task for each participant was recorded. Some tasks were inherently more difficult to complete than others and is reflected by the average time on task.
\begin{figure}[h!]
    \centering
        \includegraphics[width=\textwidth]{timeontaskscenario1}
    \caption{Time on task Scenario 1}
    \label{timeontaskscenario1}
\end{figure}
\begin{itemize}
	\item \textbf{Task 1(Start the application and set active team/repository)} showed a high average time on task. The main reason behind this was the authorization with GitHub which required users to login and authorize on the external GitHub.com page.
	\item \textbf{Task 2.1(View notifications)} showed a very similar time on task by the participants, the same can be seen on \textbf{Task 2.2(Find and archive notification)} although the time by each participant was over 80 seconds.
	\item \textbf{Task 3.1(Find reminder and perform daily reflection)} took the longest time to complete(average 222 seconds). However this was to be expected, as the daily reflection note requires participants to reflect on their work, and usually lasts for 2-5 minutes. This task also had the lartest range in completion time, ranging from 204 seconds to 272 seconds. \textbf{Task 3.2(Find and share reflection note)} and \textbf{Task 3.3(Find mood graph)} participant time on task averaged 55 and 43 seconds. 
\end{itemize}
\begin{figure}[h!]
    \centering
        \includegraphics[width=\textwidth]{timeontaskscenario2}
    \caption{Time on task Scenario 2}
    \label{timeontaskscenario2}
\end{figure}
\begin{itemize}
	\item \textbf{Task 4.1(Create a team)} showed the longest completion time in Scenario 2. The main reason behind this was the amount of data that needed to be collected from GitHub and stored in the PeacefulBanana database. The task also showed the longest range in completion time, from 185 seconds to 310 seconds. The reason behind this large gap was mainly the difference in amount of data present in the GitHub repositories they chose to create a team for. \textbf{Task 4.2(Set active team) showed no major changes in completion time,and the same can be seen in \textbf{Task 4.3(Identify team members)}}
	\item \textbf{Task 5.1(Find all milestones)} showed that 3 out of 4 participants had very similar completion times(45, 52 and 51 seconds), but the average was increased by that the last participant had a completion time of 95 seconds. \textbf{Task 5.2(Identify overdue milestones)} showed very similar completion times, while \textbf{Task 5.3(Find repository issues)} had one participant at 71 seconds, while the rest used an average of 50 seconds. \textbf{Task 5.4(Find issue \#17)} showed an average of 75 seconds, with no signifcant differences. 
	\item \textbf{Task 6.1(Generate tagcloud)} averaged on 50 seconds, \textbf{Task 6.2(Identify most used individual and team tags)} averaged 41 seconds and \textbf{Task 6.3(Find commit impact)} averaged at 36 seconds, all with no signifcant difference in completion time. 
\end{itemize}
\subsubsection{Summary of Data}
The number of errors participants made while trying to complete the tasks were captured and recorded. Critical errors leads to participant failing in completing scenario, while non-critical errors is an error that does not prevent successful completion of the scenario. These errors along with task completions and time on task average for each task are represented in Figure \ref{datasummary}.
Low completion rate, occurance of critical-errors and high time on task are highlighted in red. 
\begin{figure}[h!]
    \centering
        \includegraphics[width=\textwidth]{datasummary}
    \caption{Summary of Data}
    \label{datasummary}
\end{figure}

\subsubsection{Overall Metrics}
After completing the usability session, participants were given a \emph{System Usability Scale} form to answer\cite{brooke1996sus}. These can be seen in Figure: \ref{posttaskoverall}. \\

All participants agreed(i.e., agree or strongly agree) that they would use the application frequently and that the application was easy to use. All participants(100\%) also felt confident when using the application. The majority of the participants(75\%) felt the functions in the application were well integrated, and that most people would learn to use the system very qucikly. Half of the participants(50\%) agreed that there were inconsistencies in the system. None of the participants(0\%) found the system unnecessarily complex or that it was cumbersome to use. Further none of the participants felt users would need support of a technical person to use the system(based on the intended usergroup) or that they needed to learn a lot before getting going with the system. The participants mentioned the quickstart guide as a possible look-to document in case of trouble.  
\begin{figure}[h!]
    \centering
        \includegraphics[width=\textwidth]{posttaskoverall}
    \caption{Post task}
    \label{posttaskoverall}
\end{figure}

\subsubsection{Reflection and sharing}
\label{subsubsec:reflection}
Participants were also asked to answer the questions identified in \ref{subsec:context}
\begin{itemize}
	\item Is the application easy to use, and can users achieve their goals in a timely manner?
\end{itemize}
Feedback here was that participants were satisfied with the ease of use as can be seen above, also time-on-task show that participants were mostly quite similar in the solving of tasks, and very few spikes. 
\begin{itemize}
	\item Identify the relationship users have with the aspect of reflection and sharing personal experiences.
\end{itemize}
On this context, participants answered that reflecting on their experiences is something they often do, but they don't collect them and thus forgets exactly what the experience was about. The application helped solve this problem by prompting and allowing users to reflect and then store it for later use. \\
When it comes to sharing, all participants were positive to this, although they strongly emphasized the need for an \emph{unshare} functionality. The missing ability to unshare these notes afterhand could make them reluctant to share them in the first place, since when first shared it was always shared. One participant mentioned the possibility of letting notes be editable and share/unshareable for a specific time period, f.ex 24 hours, where afterwards they would be locked for editing.  
\begin{itemize}
	\item Does the tool present data in a way that triggers reflection for the user?
\end{itemize}
Participants responded that the tag-cloud and questions in the daily reflection note triggered them to reflect on experiences. Actually reading the questions in their mind, helped them to reflect on the experiences, instead of just having an empty text-field could lead to random thoughts beeing collected and not actually triggering reflection. The commit impact graph was mentioned as less-helpful as it didn't really justify the amount of work done. 
	
Participants also provided feedback for what they liked the most and least about the application, and recommendations for improving the application. \\
\subsubsection*{Most liked}
The participants liked the design of the application and that it was able to synchronize with their GitHub repositories automatically, without them having to worry about it. The personal tag-cloud and the ability to compare it directly with the team's tagcloud was also a joint feedback.
\subsubsection*{Least liked}
Participants commented that the way notifications were given was not optimal, and the registration process was a bit tedious.
\subsubsection{Recommendations}
In addition to the feedback gathered from section \ref{subsubsec:reflection} above this section provides recommended changes and justifications driven by the participant success rate, behaviors, and comments. The identified recommendations will improve the overall ease of use and address the areas where participants experienced problems or found the interface/information architecture unclear.
Feedback on recommendations on improvement was primarily to streamline the registration process. Also the participants commented that issues connected to a specific milestone, should be more visibly separated from issues connected to the whole repository. \\

\subsubsection*{Add unshare functionality}
\begin{figure}[h!]
    \centering
        \includegraphics[width=\textwidth]{unshareimprovement}
    \caption{Adding unshare functionality to reflection notes.}
    \label{unshareimprovement}
\end{figure}

\subsubsection*{Remove commit impact}
\begin{figure}[h!]
    \centering
        \includegraphics[width=\textwidth]{commitimprovement}
    \caption{Adding unshare functionality to reflection notes.}
    \label{commitimprovement}
\end{figure}

\subsubsection*{Task 1: Start application, login, link account with GitHub and set an active repository. } 
\begin{figure}[h!]
    \centering
        \includegraphics[width=\textwidth]{task1improvement}
    \caption{Task 1 changes and justification}
    \label{task1improvement}
\end{figure}
\subsubsection*{Task 2.2: Find the \emph{Congratulations} notification and archive it. Verify archivation.}
\begin{figure}[H]
    \centering
        \includegraphics[width=\textwidth]{task22improvement}
    \caption{Task 2.2 changes and justification}
    \label{task22improvement}
\end{figure}
\subsubsection*{Task 3.3: Find the mood-graph.} 
\begin{figure}[h!]
    \centering
        \includegraphics[width=\textwidth]{task33improvement}
    \caption{Task 3.3 changes and justification}
    \label{task33improvement}
\end{figure}
\subsubsection*{Task 5.1: Find all milestones for the current active repository.}
\begin{figure}[h!]
    \centering
        \includegraphics[width=\textwidth]{task51improvement}
    \caption{Task 5.1 changes and justification}
    \label{task51improvement}
\end{figure}
\subsubsection*{Task 5.3: Find repository related issues.} 
\begin{figure}[H]
    \centering
        \includegraphics[width=\textwidth]{task53improvement}
    \caption{Task 5.3 changes and justification}
    \label{task53improvement}
\end{figure}

\subsection{Conclusion}
% Noe mer?
Most of the participants found the PeacefulBanana application to be well-organized, comprehensive, clean and uncluttered, very useful, and easy to use. Having an application to handle reflection in their daily work is to many if not all of the participants. Implementing the recommendations and further feedback from users will ensure a continued user-friendly application.\\
While some tasks had a high time-on-task, this was due to GitHub data collection, and is not something we have control over. It is a simple request to the GitHub API, which we need to receive before continuing. 

%!TEX root = main.tex
\chapter{Expert Review}
\section{Overview}
The expert evaluation was with a senior scientist at Sintef Information and Communication Technology\footnote{Sintef ICT: \url{http://www.sintef.no/home/Information-and-Communication-Technology-ICT/}}, which also works as adjunct associate professor at NTNU. The evaluator is an expert in the field of agile software development and knowledge management in software companies. The evaluator has published several case studies of agile teamwork in the industry. He also had knowledge of GitHub and the use of it in agile development teams.  Apart from the expert review, the evaluator was never directly involved in our thesis.  \\
The evaluation performed was a type of expert walkthrough as described in \emph{Interaction Design - Beyond Human Computer Interaction}\cite{rogers2011interaction}. The evaluation we performed differs in that we also evaluate how the application can support agile software development teams and reflection through revisiting experiences. We presented the main features of the application and proceeded with a walkthrough of the application in its production state. The walkthrough consisted of typical tasks in our scenarios. After the walkthrough we had an open discussion with the evaluator on possible shortcomings or limitations, and also any advantages the evaluator had identified. We asked the evaluator to present his ideas on how our application could improve reflection in agile software development teams. We wanted an objective evaluation so we did not initially present any of our own thoughts about the application and its goals. \\
The evaluator suggested several possible limitations in the application, and also commented on how our application met some of the problems that often arise in development teams and how he could see our application potentially help limit these problems. We particularly went through the reflection workshop questions[ref her], to get feedback on the feasability of these and triggering reflection in a retrospective session. 
In addition to advantages and limitations of our applications the evaluator had some input towards related work he had seen, and how we could conduct the final evaluation in the best way. This included possible questions to ask, how to get the best possible output from the evaluation group and also some theory on how to analyze the results we got. \\
The feedback we got from the expert showed that we had proposed solutions to many of the problems he had encountered through his studies of agile teams, and specifically how to trigger better reflection. The evaluator also had some valuable feedback on possible shortcomings we had not thought of, that is an issue in a day to day working environment. 

\subsection{Overall Feedback}
The evaluation with the Sintef expert left us with the impression that the evaluator was satisified with the general functionality of the application, in terms of agile teams and reflection in these teams. 
At the point of evaluation, the application had been deployed to production state, and so most of the functionality was in place. The evaluator stated that the choices we had made on the data collection and representation of these was satisfying, as it allowed and encouraged users to reflect on their experiences, while not being too intrusive on the daily work routine. Both in aspects of individual and collaborative reflection the application and its functionality was satisfactory. As for the team aspect the evaluator saw a limitation in that it was generally hard to get new tools into the daily routine of developers. Especially the way we expect users to use \#\emph{hashtags} to tag important elements in the commit message, might take some time to work in. The evaluator expressed that he was happy with the design choices made and that we chose a web-application as a platform. This way the application is available for all individuals in the team, on a wide variety of devices. This availability is important in order to furhter lower the threshold of usage. Apart from general feedback and app-specific feedback we also got some feedback regarding the final evaluation we conducted. Especially what questions to ask, compare their previous retrospectives routines with a retrospective using our app in beforehand. Also feedback on how to properly analyze the results we would get was valuable, as the evaluator had experience in his studies, that showed possible reasons for why a specific outcome could occur. The evaluator was also pleased with the notion of allowing teams to see what notes are shared , which allowed for sharing patterns among the users. This is something that could help encourage a \emph{"discussion about reflection in the retrospective sessions"}. Another point was that individual users tends to use the same tools in different ways, so specifying what and how the application could be used before users started using the tool was important. \\ 

\subsection{App-specific feedback}
Here we will present the challenges the evaluator presented in the aspect of teamwork and reflection in agile software development teams. We will then detail the feedback from the evaluator on how potentially the PeacefulBanana application can answer these challenges. 
\subsubsection{Challenges}
\begin{figure}[h!]
    \centering
        \includegraphics[width=\textwidth]{rogerscurve}
    \caption{Roger's Innovation Adoption Curve}
    \label{rogerscurve}
\end{figure}

\begin{table}[H]
    \begin{tabularx}{\textwidth}{|l|l|X|}
    \hline
    ID & Name                & Description                                                                                                                                                                                                                                                                                                                                                    \\ \hline
    1  & Non-intrusive       & The threshold of integrating new tools into the routines of software developers is hard. The evaluator specifically referred to the \emph{Technology Adoption Curve} presented by Rogers\cite{rogers2010diffusion}, which refers to the chasm between innovators or early adopters and the early majority. This curve can be seen in Figure \ref{rogerscurve}. \\ \hline
    2  & Uniqueness          & The application should meet a demand which hasn't already been met. Also the application should provide something that a normal retrospective does not.                                                                                                                                                                                                        \\ \hline
    3  & Agile integration   & How can the application be integrated into an agile environment, helping the team to be agile and not removing the agility from the team.                                                                                                                                                                                                                      \\ \hline
    4  & Dynamic Memories    & Memories are dynamic and change over time, so there can be a lack of memorizing all important situations in a retrospective.                                                                                                                                                                                                                                   \\ \hline
    5  & Priorities          & Often agile teams develop what the developers are motivated for, and not what the customer prioritizes highest. These wrong-placed priorities can be hard to pick up.                                                                                                                                                                                          \\ \hline
    6  & Competence-overlap  & Agile teams are most efficient and deliver the highest quality work when atleast two people have the same competence, so that one can ask for help and code can be reviewed by a peer. When a developer is left alone on a piece of work, integrating these parts with the rest of the project can be an issue                                                 \\ \hline
    7  & Re-work:            & Re-doing the same piece of work is also a challenge development teams can meet. When developers constantly revisits work that already has been accepted, to make unnecessary changes, the progress of the project is slowed down. Detection of this can allow for a discussion and allowing the team to progress.                                              \\ \hline
    9  & Level of expertise: & Developers often have different levels of expertise, and different areas of expertise. Even though a developer have a high amount of impact on the code-lines commited to a project, this does not mean the others don't do important work.                                                                                                                    \\ \hline
    \end{tabularx}
    \caption {Expert review feedback}
    \label{experttable}
\end{table}
\clearpage
The challenges identified in Table \ref{experttable} was used to create a discussion around the PeacefulBanana application and how it can solve these problems. 
\textbullet{Testesen}

\subsubsection{Features}
The evaluator had some input to what additional functionality that might be implemented: 
\begin{itemize}
	\item Show parts of the source-code in the PeacefulBanana application, creating a sort of \emph{What is new?} functionality to the team. 
	\item Integrate a burndown-chart into the application. 
\end{itemize}

%  Keeping the required involvement by users of the application to a minimum was a good design choice, 
% # Feedback fra Torgeir:
% * Stille alternativer opp mot hverandre i rapporten, f.eks om de ser en nytte versus å kun bruke github. 
% * Belyser appen nye ting i forhold til vanlige retrospektiver?
% * Spørre gruppa hva de gjorde på tidligere retrospektiver, og om bruken av appen kan hjelpe. Spør de som ikke har deltatt at dersom de hadde brukt tags osv. hadde de sett nytten? Og da evt til hvilken grad.  
% * Spør: Hvorfor har de ikke brukt det? Tid? Features? Vanskelig? Var det i veien?

% ## Fordeler
% * Non-intrusive: Veldig bra
% * Er unikt: Han hadde ikke sett noe lignende verktøy
% * Er et kjent problem at man ikke husker alt, notes kan hjelpe men det kan bli mye data å analysere som går litt imot smidig. Viktig spørsmål å se om man husker mer ved bruk av appen, altså om man går glipp av noe uten å bruke appen. 
%* Finne feilprioriteringer, ofte er det feilvurderinger av hva som er viktig å gjøre. Oftest gjøres det som utviklerne vil og ikke kunden. Med appen kan man se tendenser til mye jobbing på feil ting. Samt. at man ser at milestones går overdue. 
%* Koble ting opp mot teorimodell: Kompetanseoverlapp. Sitter noen mye alene? Tagclouden kan vise slike tendenser.
%* Re-work: Oppdage dette, vi har muligheter til å oppdage mye gjenåpninger på issues og forklare dette. Mye jobbing med samme ting kan oppdages i appen. 

%## Utfordringer
%* Får man noe nyttig utav det
%* Få personer til å faktisk bruke tags i hverdagen. 
%* Generelt vanskelig terskel å få folk til å bruke verktøy. -> Rogers curve-diagram. 
%* Det vil være folk som ikke vil bidra like mye, early adopters vs laggers. 
%* Forskjellig nivå på utviklere. Noen vil dominere commit statistikken, men betyr det da at de er viktigere enn de andre? (Som kanskje gjør andre viktige ting i prosjektet). Det kan gi feil totalbilde. 

%## Features
%* Ha deler av kildekoden i peaceful, en slags "hva har skjedd" feature. Overlapper litt med github
%* Burndown chart. Har github noe lignende kanskje? sjekk

%## Studier
%* Vitner husker ofte feil i rettsaker -> hukommelse er ikke bombesikker/til å stole på alltid. Derfor lurt å ta ferske erfaringer

%### Related Work:
%* Hackystat : https://code.google.com/p/hackystat/ 
\section{Focus Group}
Focus groups is a form of qualitative research which can be compared to semi-structured group interviews\citep{rogers2011interaction}. Participants in a group are asked about their opinions and views towards a product or concept based on their background and experiences\citep{krueger2008focus}. The group should consist of six to ten persons, where participants are not inhibited to present their honest opinions and experiences\citep{krueger2008focus}. The discussion is governed by a facilitator which have the task of keeping focus on the discussion in order to get answers to the questions that have been prepared beforehand\citep{krueger2008focus, nielsen1997use}. The facilitator is responsible for keeping the discussion open, uninhibited, non-judgmental and also making sure that all participants are allowed to present their views\citep{powell1996focus}. 

A focus group usually lasts between 90 and 120 minutes, and is conducted in a neutral place. Limitations of a focus group include that you only get information on what potential users say, and not what they do or if it aligns with the reality\citep{nielsen1997use}. Another limitation is that users often think they need something else than what they really need. Therefore it is important to demonstrate something concrete, where users understand fully what the application is, what it does and what it achieves. 
The theory above and the guide proposed by \citet{FocusGrpGuide} provided us with a basis on how to conduct a focus group interview. We gave the participants a smooth and snappy introduction to the agenda of the interview.

\subsection{Focus group context}
The application was evaluated with a focus group consisting of 8 NTNU students, working together in a group in the IT2901 bachelor - project. The group were using scrum as their agile project methodology, and have also used it previously in development projects. A group of 8 is fairly big, although larger focus groups are recommended in order to collect more commentaries and details from discussions\citep{morgan1998planning}. Participants in the focus group were familiar with the use of GitHub and were also using GitHub for their project at the time of evaluation. In their project they use retrospectives after each sprint, since they are using scrum. This provides the focus group with participants eager to improve their collaboration in agile teams and to improve reflection, both individually and in teams during retrospective sessions. The focus group was hosted at NTNU, in a private workshop lab with a circular table.  

The group were given the reflection scale from the MIRROR evaluation toolbox before starting the evaluation. Their answers and relationship to reflection can be seen in table \ref{reflectionscaleresults}. 
\begin{figure}[H]
\centering
	\includegraphics[width=\textwidth]{reflectionscaleresults}
\caption{Reflection scale results}
\label{reflectionscaleresults}
\end{figure}
The group generally agree with the importance of reflection overall in the reflection scale questionnaire, however they disagree a bit in regards to team-reflection. This is quite interesting, but during the interview we noticed a few very dominant figures in the team and this might be the reason why the answers vary a bit regarding team-reflection.

\subsection{Ground Rules}
In the introduction we identified a set of ground-rules that would make the interview go smoother and keep the participants distraction free. Most of these are based on the ground rules defined in \citet{FocusGrpGuide}, and can be seen in the list below.

\begin{enumerate}
\label{groundrulestest}
\item Mobile phones are to be turned off during the entire interview. If you are not able to do so and receive a call ,please do this as quietly as possible.
\item We are recording the session on tape, so only one person should speak at a time.
\item There is no right or wrong answer here, only opinions.
\item You do not need to agree with others, but you must listen respectfully as others share their views.
\item Talk to each other, not directly to the evaluator.
\end{enumerate}

\subsection{Data Collection and Analysis}
The session was recorded, in order to analyze the results after hand and also to be able to participate actively during the walkthrough. We had prepared some application-specific questions to ask the participants during the session. These questions consisted of the most relevant app-specific questions from Section 9.1.3 in the MIRROR evaluation toolbox \citep{mirrorevaluation}. These questions were slightly modified to be used as part of the focus group and can be seen in Table: \ref{questiontable}
\begin{table}[H]
    \begin{tabularx}{\textwidth}{|l|X|}
    \hline
    1  & Can PeacefulBanana help you to collect information relevant to reconstructing experiences from work? \\ \hline
    2  & Can the application help you to reflect on experiences from work?                                    \\ \hline
    3  & Can the application help you to capture reflection outcomes?                                         \\ \hline
    4  & Does the application help you by making daily reflection notes available for later use?              \\ \hline
    5  & Does the application remind you to reflect on experiences?                                           \\ \hline
    6  & Does the application help you to find relevant experiences from others in your team?                 \\ \hline
    7  & Does the application help you to remember previous experiences and reflections?                      \\ \hline
    8  & Does the application help you to store information regarding work experiences?                       \\ \hline
    9  & Can the application help you to decide if and when to reflect?                                       \\ \hline
    10 & Do the application help you by supporting the sharing of experiences?                                \\ \hline
    11 & Does the application guide you on how to share experiences with others?                              \\ \hline
    12 & Can the application improve your collaboration?                                                      \\ \hline
    13 & Does the application provide relevant content for reflection?                                        \\ \hline
    14 & Did the application guide you through the reflection process?                                        \\ \hline
    \end{tabularx}
    \caption{}
    \label{questiontable}
\end{table}

After a brief presentation of the application itself and our goal with this thesis, the walkthrough started. The goal of the evaluation was primarily to see how an agile development team could integrate our tool into their daily routine. The application was evaluated with test-data, so participants could easily see how it would function in a daily work environment. After showcasing the different features, the researchers facilitated the focus group discussion on how the application could promote reflection for the team, but and also potential shortcomings or challenges to integrating it in their daily routine. The questions asked during the focus group were largely open-ended, which allowed participants freedom to express their views on the application\citep{yin2008case}. The focus group was conducted in a responsive manner, allowing us to follow up on issues uncovered mid-session and adjust the content of the focus group based on this\citep{rubin2011qualitative, wengraf2001qualitative}.

While one researcher facilitated the session, another observed and took notes, with timestamps of important parts. We swapped roles during the session, in order to prevent any variance in the notes and questioning. Any ambiguity was clarified with the participant before moving on. In order to aid analysis, the focus group was recorded and transcribed, before being analyzed. The group was also filmed so that their body-language could give us an indication about their involvement in the discussion and it was pointed out by the researchers that the interview would be anonymized to ensure that the statements could not be linked back the them. Quotes from participants are denoted with \emph{FGX}, which stands for \emph{Focus Group user X}. This was done in order to keep participants anonymous. 

\subsection{Why they did not use it?}
The group was intended to participate in a case-study evaluation of the system over a period of time, but expressed that the stress level they where under during the test period and the amount of work they had remaining, made them focus on that rather than testing the tool for us. They also expressed that since the tool was made available for them so late in the process and the fact that they already had a routine worked in, they often forgot to include the tool in their routine on a daily basis. Additionally the group only got together and worked a couple of days a week, was a bit of a surprise to us, as we where expecting them to work almost full-time on the project.

\subsection{If they would have used it}
When discussing \emph{tags}, the group stated that they felt this might become \emph{'off topic'} and that team-members could use tags not relevant to the work at hand. This could in turn corrupt the tag clouds since \emph{'off topic'} tags would gain magnitude if people used the same tags for whatever they commited. One of the participants expressed that he would have liked the possibility to tag an entire commit with a theme like [GUI] or [BACKBONE], which were something the other participants felt would be a great addition to the existing possibilities regarding tags.
\begin{quote}
\emph{Adding categories in brackets or something like that would put the hash tags in perspective too what people are working on and see how much time of their day goes to what part of the project. (FG1)}
\end{quote}
This would allow tags to be categorized into different parts of the project, which we feel could add another dimension to the tool and help the team see more relevance in the tags used. However we feel that the themes or categories should be decided at project-/iteration-start, or at least agree on some rules for what tags can be used on what piece of work. The findings discussed implies that the group answered yes to question 2, 3 and 8 in table \ref{questiontable} with the addendum described above as a possible feature improvement. 

The group also had a few pointers regarding the tag cloud. The group felt tag-clouds was a good representation, the relationship between the team tag-cloud and the individual tag-cloud was not easily identified. They proposed that tags needed to be placed at the same place and be in the same color in the tag-cloud, in order to allow for easier comparison. In the demonstration, the tags were neither placed in the same place or in the same color. 
\begin{quote}
\emph{It is hard to locate the same tags in both of the tag clouds when they are placed in different regions of the cloud and in different colors. (FG2)}
\end{quote}
It was also debated whether or not it would make any difference to weight tags based on the amount of work behind the commit. We all agreed that this could be a feature that could be implemented later. Some of the group members pointed out that even only one line changed could be just as important as an entire class being implemented.
\begin{quote}
\emph{One line to fix a bug, can be as important or even more important than a entire new class. (FG2)}
\end{quote}

The group mentioned that while doing their retrospectives, they struggled to remember what they had done the previous weeks. When discussing daily reflection notes, they stated that this would give them an new dimension to ordinary retrospectives with the possibility to generate questions based on the work that had been done each day. This way, the most important work would be captured each day and easily remembered. The group also had some comments on how to possibly improve the feature: By adding a text field where the user could explain what he had done that day, might enhance the level of reflection. This was something they felt was missing from project-management tools like Trello\footnote{\url{http://www.trello.com}} which they were already using. When we introduced the group to the feature \emph{workshop preparation} we were overwhelmed by the positive response, but they also had some input on how it could be improved. The group stated that we should include the team-wide tag-cloud for the selected period. These features mentioned gave the group the possibility to reconstruct experiences recorded earlier and thus answering questions 1, 7 and 13 in table \ref{questiontable}.

As the group got closer and closer to their final evaluation they experienced that their retrospectives got shorter and shorter, it also lacked structure. This corresponds with the findings described by \citep{kasi2008post} in Section \ref{cha:introduction}. When discussing the workshop-feature some of the group members commented that it could be great for a project manager that does not spend that much time with the code hands-on and in general a great tool for retrospectives, but since they had not tested the tool in they way they conduct work, there was no way for them to verify it. This suggests that the group are positive to question 14 in table \ref{questiontable}.

\begin{quote}
\emph{I would imagine that this would be very useful for a project manager that does spend all day hands-on with coding. (FG1)}
\end{quote}
\subsection{Comments}
The group had some comments and suggestions on how to improve the prototype. As for the daily reflection note feature, a suggestion was that instead of having the note non-editable and allow for only one note each day, users could edit the note whenever they wanted during the day and at the end of it, the note will be locked for user editing. This suggestion would give the user the option of adding to a note all day long. The downside of the change would be that the freshness of the reflection would be mitigated. When the reflection on that days experiences have taken place, allowing users to go back and change their input would take some of the value out of the reflection. The group suggested that the daily reflection note should also include a team tag-cloud, in addition to the commit impact cake-diagram and the individual tag-cloud.

The group also mentioned that the current notifications were not catching their attention, and suggested they be should be more dominating on the site, making the notifications impossible to miss. As they stated: 
\begin{quote}
\emph{As notifications are there to remind users to do their daily reflection, missing the notification might lead to missing the daily reflection also.}
\end{quote}
And:
\begin{quote}
\emph{The current notification is not invasive enough, it should be impossible to ignore. (FG3)}
\end{quote}
This indicates that the group slightly disagreed to question 5 in Table \ref{questiontable}, as they wanted the reminder to be more dominant and impossible to avoid.

It was also suggested that the group could have a meeting where they planned themes and tags to be used for the next iteration, in order to make the tool more \emph{scrum specific}, as they put it, in terms of project management. Making the tool less invasive in the Scrum process is something we feel might ensure that users visit the tool more often, as it's not interfering as much with their daily routine. 

One group member said it would be interesting to see if there is a connection between what the users are working on and their mood. I.e. what are users working on when they are in a bad mood or if a user is always unhappy when working with the graphical design, he should probably not work more on that part of the project.
\begin{quote}
\emph{Seems like a natural and good tool. (FG2)}
\end{quote}

\section{Discussion}
In this section we will discuss the results and experiences gathered during the usability test, expert review and focus group evaluation of the PeacefulBanana application. We will discuss this in regard of our research questions PeacefulBanana were intended to answer. We will answer all the sub research questions first, since these are more specific by trying to answer parts of the main research question. Finally we will summarize the discussion by answering the main research question, since this is the most general in terms of reflection and learning from experiences. 

\subsection{Sub RQ1}
\noindent\makebox[\linewidth]{\rule{\textwidth}{0.5pt}} 

\begin{center}
    How to scaffold collection of data in order to promote reflection? 
\end{center}

\noindent\makebox[\linewidth]{\rule{\textwidth}{0.5pt}} 
The expert evaluator expressed a concern regarding the amount of data that needed to be processed and that it should not stop the team from being agile, but also pointed out that the feature itself could help during a retrospective session. He also pointed out that the daily reflection note provided the users with a quick and easy way to reflect on a daily basis and that this would work for both users in teams and individuals. However individuals would use the functionality differently than team-users. Team users would also have an effect by looking at the me-vs-team tag-cloud, and he pointed out that this could trigger reflection as well when comparing your own work with the rest of the team.

The focus group pointed out that they might wanted more data showed or at least the possibility to view data in their original form\footnote{Entire commit messages and issues titles in the tag cloud f.ex.} in additions to data we provided for them. They expressed that the use of commit messages to generate questions for retrospectives would give them a another dimension and more structure.

\subsection{Sub RQ2}
\noindent\makebox[\linewidth]{\rule{\textwidth}{0.5pt}} 

\begin{center}
How to increase the tendency to reflect on experiences, both individually and as a team? \\
\end{center}  

\noindent\makebox[\linewidth]{\rule{\textwidth}{0.5pt}}
Regarding the individually reflection the focus group expressed that a reminder in a tool like PeacefulBanana is a good way to trigger reflection, however they felt like the notification was a bit mild and should be more invasive to more force them to reflect over the last the 24 hours of work. Some group members expressed that the fact that the notifications needs to be read and that you can have multiple of them defeats the purpose of the notification, they are not interested in a historic overview of reflection reminders. 

They felt that the tool did not provide them with any reminder in order to improve their reflection rate as a team, however some of them stated that the link only visible to team owner and manager could have some effect in order for them to plan a retrospective session.

\subsection{Sub RQ3}
\noindent\makebox[\linewidth]{\rule{\textwidth}{0.5pt}} 

\begin{center}
How to bring together contributions from multiple users, and sharing these in a collaborative environment? \\
\end{center}

\noindent\makebox[\linewidth]{\rule{\textwidth}{0.5pt}}
The focus group liked that the tool had the possibility to share notes with other team members and that these notes can be inspected. The expert evaluator stated that this was vital for the users to reconstruct work experiences in order for them to reflect over past experiences. The tag-cloud could also be used to identify how much you work on different issues.

The expert reviewer stated that the application introduced some new aspects that changed the normal routine for a retrospective, by generating suggestive questions based on the work done. He then expressed that it would have liked the application to be tested with a real agile software development team over a period of time and then conduct a retrospective session with that team. It would then be interesting to see if the laggers described in Figure \ref{rogerscurve}(Section \ref{subsec:overallfeedback}) would have as much to contribute with as the early adopters.

\subsection{Main RQ}
\noindent\makebox[\linewidth]{\rule{\textwidth}{0.5pt}} 

\begin{center}
How to promote experienced-based reflection based on project artifacts collected from version-control systems? \\
\end{center}

\noindent\makebox[\linewidth]{\rule{\textwidth}{0.5pt}} 
During the evaluation every step of the PeacefulBanana work-process have been evaluated by the different evaluators and is described in the sections above when discussing the different sub research questions. The first step is collect and is related to collecting data for reflection, here the application gathers data from GitHub. Then the data is analyzed and the tags added by the team members are added to each member and displayed in their daily as well as the overall team cloud of tags. These tags are placed here to remind the user on what the have been working on the last day, during both evaluations it was expressed that this was something the users would benefit greatly from.

The next step in the process is then reflection, here we use the data collected in the step above to trigger reflection when showing the user the me-vs-team tag-cloud. When the user sees the tags he has been working on compared with the rest of the team this can trigger reflection. During the evaluations there where some minor comments regarding placements of the tags, but after discussing it with the focus group we all agreed that this was something that could indeed trigger reflection. The users also have to fill out a daily reflection note and this process is more scaffolded, since the users have to full out a scheme with a set of questions related to their daily activities.

The third step consists of sharing your experiences with team members in order for them to learn from them, here the users can both share their daily reflection notes for the other team members. Or the issues described in them can be discussed in the retrospective reflection session scheduled to take place after each iteration.

The fourth and last step consists of taking the outcomes of your reflection and putting it to use in work related tasks in order to improve.

The evaluators expressed both concerns and appreciation for the use of these data and the fact that the expert reviewer would have liked to see a study with the application is a great sign that the he find the tool useful for agile teams.
% Reflectere over alle de overnevte punktene over og hvordan disse svarer på 