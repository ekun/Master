%!TEX root = main.tex
\chapter{Evaluation}
In this chapter we will describe how we evaluated the PeacefulBanana tool, and how it fulfills the requirements.

% This chapter evaluates your prototype and how it fulfills the
% stated requirements. Depending on the focus and on available resources
% the evaluation might involve users. If you do not have resources for a user
% evaluation, you can have a scenario that helps illustrate how the prototype
% fulfills the requirements and discuss limitations.

\section{Usability test}

\section{Expert Review}
%!TEX root = main.tex
\chapter{Expert Review}
\section{Overview}
The expert evaluation was with a senior scientist at Sintef Information and Communication Technology\footnote{Sintef ICT: \url{http://www.sintef.no/home/Information-and-Communication-Technology-ICT/}}, which also works as adjunct associate professor at NTNU. The evaluator is an expert in the field of agile software development and knowledge management in software companies. The evaluator has published several case studies of agile teamwork in the industry. He also had knowledge of GitHub and the use of it in agile development teams.  Apart from the expert review, the evaluator was never directly involved in our thesis.  \\
The evaluation performed was a type of expert walkthrough as described in \emph{Interaction Design - Beyond Human Computer Interaction}\cite{rogers2011interaction}. The evaluation we performed differs in that we also evaluate how the application can support agile software development teams and reflection through revisiting experiences. We presented the main features of the application and proceeded with a walkthrough of the application in its production state. The walkthrough consisted of typical tasks in our scenarios. After the walkthrough we had an open discussion with the evaluator on possible shortcomings or limitations, and also any advantages the evaluator had identified. We asked the evaluator to present his ideas on how our application could improve reflection in agile software development teams. We wanted an objective evaluation so we did not initially present any of our own thoughts about the application and its goals. \\
The evaluator suggested several possible limitations in the application, and also commented on how our application met some of the problems that often arise in development teams and how he could see our application potentially help limit these problems. We particularly went through the reflection workshop questions[ref her], to get feedback on the feasability of these and triggering reflection in a retrospective session. 
In addition to advantages and limitations of our applications the evaluator had some input towards related work he had seen, and how we could conduct the final evaluation in the best way. This included possible questions to ask, how to get the best possible output from the evaluation group and also some theory on how to analyze the results we got. \\
The feedback we got from the expert showed that we had proposed solutions to many of the problems he had encountered through his studies of agile teams, and specifically how to trigger better reflection. The evaluator also had some valuable feedback on possible shortcomings we had not thought of, that is an issue in a day to day working environment. 

\subsection{Overall Feedback}
The evaluation with the Sintef expert left us with the impression that the evaluator was satisified with the general functionality of the application, in terms of agile teams and reflection in these teams. 
At the point of evaluation, the application had been deployed to production state, and so most of the functionality was in place. The evaluator stated that the choices we had made on the data collection and representation of these was satisfying, as it allowed and encouraged users to reflect on their experiences, while not being too intrusive on the daily work routine. Both in aspects of individual and collaborative reflection the application and its functionality was satisfactory. As for the team aspect the evaluator saw a limitation in that it was generally hard to get new tools into the daily routine of developers. Especially the way we expect users to use \#\emph{hashtags} to tag important elements in the commit message, might take some time to work in. The evaluator expressed that he was happy with the design choices made and that we chose a web-application as a platform. This way the application is available for all individuals in the team, on a wide variety of devices. This availability is important in order to furhter lower the threshold of usage. Apart from general feedback and app-specific feedback we also got some feedback regarding the final evaluation we conducted. Especially what questions to ask, compare their previous retrospectives routines with a retrospective using our app in beforehand. Also feedback on how to properly analyze the results we would get was valuable, as the evaluator had experience in his studies, that showed possible reasons for why a specific outcome could occur. The evaluator was also pleased with the notion of allowing teams to see what notes are shared , which allowed for sharing patterns among the users. This is something that could help encourage a \emph{"discussion about reflection in the retrospective sessions"}. Another point was that individual users tends to use the same tools in different ways, so specifying what and how the application could be used before users started using the tool was important. \\ 

\subsection{App-specific feedback}
Here we will present the challenges the evaluator presented in the aspect of teamwork and reflection in agile software development teams. We will then detail the feedback from the evaluator on how potentially the PeacefulBanana application can answer these challenges. 
\subsubsection{Challenges}
\begin{figure}[h!]
    \centering
        \includegraphics[width=\textwidth]{rogerscurve}
    \caption{Roger's Innovation Adoption Curve}
    \label{rogerscurve}
\end{figure}

\begin{table}[H]
    \begin{tabularx}{\textwidth}{|l|l|X|}
    \hline
    ID & Name                & Description                                                                                                                                                                                                                                                                                                                                                    \\ \hline
    1  & Non-intrusive       & The threshold of integrating new tools into the routines of software developers is hard. The evaluator specifically referred to the \emph{Technology Adoption Curve} presented by Rogers\cite{rogers2010diffusion}, which refers to the chasm between innovators or early adopters and the early majority. This curve can be seen in Figure \ref{rogerscurve}. \\ \hline
    2  & Uniqueness          & The application should meet a demand which hasn't already been met. Also the application should provide something that a normal retrospective does not.                                                                                                                                                                                                        \\ \hline
    3  & Agile integration   & How can the application be integrated into an agile environment, helping the team to be agile and not removing the agility from the team.                                                                                                                                                                                                                      \\ \hline
    4  & Dynamic Memories    & Memories are dynamic and change over time, so there can be a lack of memorizing all important situations in a retrospective.                                                                                                                                                                                                                                   \\ \hline
    5  & Priorities          & Often agile teams develop what the developers are motivated for, and not what the customer prioritizes highest. These wrong-placed priorities can be hard to pick up.                                                                                                                                                                                          \\ \hline
    6  & Competence-overlap  & Agile teams are most efficient and deliver the highest quality work when atleast two people have the same competence, so that one can ask for help and code can be reviewed by a peer. When a developer is left alone on a piece of work, integrating these parts with the rest of the project can be an issue                                                 \\ \hline
    7  & Re-work:            & Re-doing the same piece of work is also a challenge development teams can meet. When developers constantly revisits work that already has been accepted, to make unnecessary changes, the progress of the project is slowed down. Detection of this can allow for a discussion and allowing the team to progress.                                              \\ \hline
    9  & Level of expertise: & Developers often have different levels of expertise, and different areas of expertise. Even though a developer have a high amount of impact on the code-lines commited to a project, this does not mean the others don't do important work.                                                                                                                    \\ \hline
    \end{tabularx}
    \caption {Expert review feedback}
    \label{experttable}
\end{table}
\clearpage
The challenges identified in Table \ref{experttable} was used to create a discussion around the PeacefulBanana application and how it can solve these problems. 
\textbullet{Testesen}

\subsubsection{Features}
The evaluator had some input to what additional functionality that might be implemented: 
\begin{itemize}
	\item Show parts of the source-code in the PeacefulBanana application, creating a sort of \emph{What is new?} functionality to the team. 
	\item Integrate a burndown-chart into the application. 
\end{itemize}

%  Keeping the required involvement by users of the application to a minimum was a good design choice, 
% # Feedback fra Torgeir:
% * Stille alternativer opp mot hverandre i rapporten, f.eks om de ser en nytte versus å kun bruke github. 
% * Belyser appen nye ting i forhold til vanlige retrospektiver?
% * Spørre gruppa hva de gjorde på tidligere retrospektiver, og om bruken av appen kan hjelpe. Spør de som ikke har deltatt at dersom de hadde brukt tags osv. hadde de sett nytten? Og da evt til hvilken grad.  
% * Spør: Hvorfor har de ikke brukt det? Tid? Features? Vanskelig? Var det i veien?

% ## Fordeler
% * Non-intrusive: Veldig bra
% * Er unikt: Han hadde ikke sett noe lignende verktøy
% * Er et kjent problem at man ikke husker alt, notes kan hjelpe men det kan bli mye data å analysere som går litt imot smidig. Viktig spørsmål å se om man husker mer ved bruk av appen, altså om man går glipp av noe uten å bruke appen. 
%* Finne feilprioriteringer, ofte er det feilvurderinger av hva som er viktig å gjøre. Oftest gjøres det som utviklerne vil og ikke kunden. Med appen kan man se tendenser til mye jobbing på feil ting. Samt. at man ser at milestones går overdue. 
%* Koble ting opp mot teorimodell: Kompetanseoverlapp. Sitter noen mye alene? Tagclouden kan vise slike tendenser.
%* Re-work: Oppdage dette, vi har muligheter til å oppdage mye gjenåpninger på issues og forklare dette. Mye jobbing med samme ting kan oppdages i appen. 

%## Utfordringer
%* Får man noe nyttig utav det
%* Få personer til å faktisk bruke tags i hverdagen. 
%* Generelt vanskelig terskel å få folk til å bruke verktøy. -> Rogers curve-diagram. 
%* Det vil være folk som ikke vil bidra like mye, early adopters vs laggers. 
%* Forskjellig nivå på utviklere. Noen vil dominere commit statistikken, men betyr det da at de er viktigere enn de andre? (Som kanskje gjør andre viktige ting i prosjektet). Det kan gi feil totalbilde. 

%## Features
%* Ha deler av kildekoden i peaceful, en slags "hva har skjedd" feature. Overlapper litt med github
%* Burndown chart. Har github noe lignende kanskje? sjekk

%## Studier
%* Vitner husker ofte feil i rettsaker -> hukommelse er ikke bombesikker/til å stole på alltid. Derfor lurt å ta ferske erfaringer

%### Related Work:
%* Hackystat : https://code.google.com/p/hackystat/
%Evaluate results from expert review. % Do we have any expert reviews?

\section{Focus group}
The prototype was evaluated with a focus group consisting of 8 NTNU students, working together in a group in the IT2901 bachelor - project. The group were using scrum as their agile project methodic, and have also used it previously in development projects. A group of 8 is fairly big, although larger focus groups are recommended in order to collect more commentaries and details from discussions\cite{morgan1998planning}. Participants in the focus group were familiar with the use of GitHub and were also using GitHub for their project at the time of evaluation. In their project they use retrospectives after each sprint, since they are using scrum. This provides the focus group with participants eager to improve their collaboration in agile teams and to improve reflection, both individually and in teams during retrospective sessions. The focus group was hosted at NTNU, in a private workshop lab. 

The group were given the reflection scale from the MIRROR evaluation toolbox before starting the evaluation. Their answers and relationship to reflection can be seen in table \ref{mirrormodel}. 
\begin{figure}[H]
\centering
	\includegraphics[width=\textwidth]{reflectionscaleresults}
\caption{Reflection scale results}
\label{mirrormodel}
\end{figure}
% Maybe nevne noen av resultatene her?

\subsection{Data Collection and Analysis}
The structure of the focus group was a walkthrough of the application and an open discussion during the walkthrough. The session was recorded, in order to analyze the results afterhand and also be able to participate actively during the walkthrough. We had prepared some application-specific questions to ask the participants during the session. These questions consisted of the most relevant app-specific questions from Section 9.1.3 in the MIRROR evaluation toolbox\cite{mirrorevaluation}. These questions were slightly modified to be used as part of the focus group and can be seen in Table: \ref{questiontable}
\begin{table}[H]
    \begin{tabularx}{\textwidth}{|l|X|}
    \hline
    1  & Can PeacefulBanana help you to collect information relevant to reconstructing experiences from work? \\ \hline
    2  & Can the application help you to reflect on experiences from work?                                    \\ \hline
    3  & Can the application help you to capture reflection outcomes?                                         \\ \hline
    4  & Does the application help you by making daily reflection notes available for later use?              \\ \hline
    5  & Does the application remind you to reflect on experiences?                                           \\ \hline
    6  & Does the application help you to find relevant experiences from others in your team?                 \\ \hline
    7  & Does the application help you to remember previous experiences and reflections?                      \\ \hline
    8  & Does the application help you to store information regarding work experiences?                       \\ \hline
    9  & Can the application help you to decide if and when to reflect?                                       \\ \hline
    10 & Do the application help you by supporting the sharing of experiences?                                \\ \hline
    11 & Does the application guide you on how to share experiences with others?                              \\ \hline
    12 & Can the application improve your collaboration?                                                      \\ \hline
    13 & Does the application provide relevant content for reflection?                                        \\ \hline
    14 & Did the application guide you through the reflection process?                                        \\ \hline
    \end{tabularx}
    \caption{}
    \label{questiontable}
\end{table}

After a brief presentation on the application itself and our goal for this thesis, the walkthrough started. The goal of the evaluation was primarily to see how an agile development team could integrate our tool into their daily routine. The application was evaluated with test-data, so participants could easily see how it would function in a daily work environment. After showcasing the different features, the researchers facilitated the focus group discussion on how the application could promote reflection for the team, but and also potential shortcomings or challenges to integrating it in their daily routine. The questions asked during the focus group were largely open-ended, which allowed participants freedom to express their views on the application\cite{yin2008case}. The focus group was conducted in a responsive manner, allowing us to follow up on issues uncovered mid-session and adjust the content of the focus group based on this\cite{rubin2011qualitative, wengraf2001qualitative}.\\
While one researcher facilitated the session, another observed and took notes, with timestamps of important parts. We swapped roles during the session, in order to prevent any variance in the notes and questioning. Any ambiguity was clarified with the participant before moving on. In order to aid analysis, the focus group was recorded and transcribed, before being analyzed. 

\subsection{Why they did not use it?}
The group was intended to test use the system over a period of time, but expressed that the stresslevel they where under during the test periode and the ammount of work they had remaining made them focus on that rather than testing the tool for us. They also expressed that since the tool was made availible to them so late in the process and the fact that they already had a rutine worked in, they therefor forgot to include the tool in that rutine on a day to day basis. The fact that they only worked a couple of days a week, which was a bit of an surprise to us as we where expecting them to work almost fulltime on the project.

\subsection{If they would have used it}
When discussing tags, the group stated that they felt this might become 'off topic' and that team-members could use tags not relevant at all. This would the corrupt the tagclouds since 'off topic' tags would gain magnitude since people could use the same tags for what ever they commited, but when tagging issues f.ex. However one of the members expressed that he would have liked the possibility to tag an entire commit with a theme like [GUI] or [BACKBONE], the other members felt like this would been a great addition to the excisting possibilities with tags. This would then add the posibility to categories the tags in to different part of the project and we feel that this could add another dimension to the tool and help the team to see more relevance in the tags, however we feel that the themes should be decide at project-/iteration-start or atleast to scaffold the themes some. This implies that the group answeres yes to question 2, 3 and 8 in table \ref{questiontable} with the addendum described above which would improve the feature.

The group got had a few pointers regarding the tagcloud as well, even though the felt that it would give a good representation they felt that in order for the users to see the the relationship between the teams tagcloud and their own, tags needed to be placed at the same place and be in the same color for comparison. In the demonstration the tags where neither placed in the same place nor in the same color. It was also debated wheter or not it would make any difference to weight tags based on the amount of work behind the commit, but we all agreed that this could be a feature that could be implemented later. Some of the group members pointed out that even only one line changed could be just as important than an entire class. 

While doing their retrospektives, they struggled to remember what they have done the week before and when discussing daily reflection notes. They felt like this would give them an new dimension to ordinary retrospeives, however they had some comments on how to possibly improve the feature. By adding a textfield where the user could explain what he had done that day it might enhance the level of reflection. This was something they felt was missing from project-management tools like Trello\footnote{\url{http://www.trello.com}} as they where using. When we introduced the group to the feature we call workshop preparation we where overwhelmed by the response, but they felt it could have been even better if it would have included the total tagcloud for the periode selected. This gave them a possibility to reconstruct experiences recorded earlier and thus answering the question 1 in table \ref{questiontable}.

As the group got closer and closer to their final evaluation they experienced that their retrospektives got shorter and shorter, it also lacked structure. When discussing the workshop-feature some of the group members commented that this could be greate for a project manager that does not spend that much time with the code hands-on and in general a great tool for retrospectives, but since they had not tested the tool their was no way for them to verify it.

\subsection{Comments}
%Generelle kommentarer gruppen hadde til appen
The group had some comments / suggestions on how to improve the prototype, there was one of the suggestions that we felt was essential to improve the daily reflection note feature. Instead of having the note non editable and only one each day, they suggested that we gave the users the possibility to edit the note all day and at the end of the day it will be locked so that the user can not edit it any more. This suggestion would give the user the option of noting all day long, but they will miss the feature of relfection on the days work at the end of the day as the feature was designed to do. They also suggested that the summary\footnote{With a impact cake-diagram and your own tagcloud.} should include a team tagcloud and the commits done by the user for that day.

They felt like the current notification with the daily reflection note was not enough to grab their attention and should possibly be more dominating on the site so that it was impossible to miss. This indicates that the group slightly disagreed to the question 5 as they wanted the reminder to be more dominant and impossible to avoid.

It was also suggested that the group could have a meeting where they planed themes and tags for that iteration so that the tool would become more 'scrum-friendly' as they put ut, when it comes to project-management. 

One of the group members said that it would be interessting to see if there is a connection on what the users are working on when they are in a bad mood, for example if a user is allways unhappy when working on the GUI he should probably not work more on that part of the project.
