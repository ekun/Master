\chapter{Implementation}
In many cases, you will not be able to realize the full design. Often the implementation is only a demonstrator of ideas. 
It is therefore important that you focus on the most important aspects of your system (depending on research questions). 
In the report you have to justify the choices that are done.

Milestones:
The milestone feature in Github feature the possibility of setting up the different deliveries, iterations etc.

Commits:
The tool can retrieve all commits done by a user, or even all commits for a given project/repository. This includes commit comments, timestamps and other relevant data. This can be used together with milestones to see commit activity, commit spikes and such. We can also through comments etc. see if there were any specific problems during a certain period. This will be tightly integrated with the milestones set up in the github project. 

Tags/Labels:
Git(the version control system which Github is based on) also allows tagging specific points in history with labels. Since this is different from the tagging we had in mind we need to provide our own tagging methods to produce tagclouds. Our approach is based on the way twitter uses tagging(hashtags), but there might be useful to show git specific tags as well.

Repositories
A projects artifact base is stored as repositories on Github. We can extract this data.
