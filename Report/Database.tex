%!TEX root = main.tex
\newpage
\thispagestyle{empty}
\mbox{}
\chapter{PeacefulBanana Database Descriptions}
This part of the appendix will describe a detailed walkthrough of each class in PeacefulBanana. 
\section{User data}
The first section is devoted to show and tell all data related to users, the tables described represents the different data types associated with users. Users are a vital part of the implementation and everything is centered around these users, it is therefore important to store as much relevant data as possible. \\

\subsection*{User}
This table hold data related to each user and only the data which is unique to them.
The password are encrypted with SHA-512 encryption which is a part of SHA-2 a set of encryption hashing functions created by the U.S. National Security Agency in 2002 and is a known not to be reversal.\\

\vspace{0.5cm}
\begin{tabularx}{\linewidth}{| c | X |}
    \hline
    \rowcolor[gray]{0.8}
    \textbf{Field} & \textbf{Description} \\
    \hline
    Id & Unique id generated when the user is first created\\ \hline
    UserName & This field stores the users username\\ \hline
   	Password & This field stores the users password as a encrypted string, the password is encrypted with SHA-512 encryption\\ \hline
    Email & This field stores the users email\\ \hline
    FirstName & This field stores the users first name\\ \hline
    LastName & This field stores the users last name\\ \hline
    SelectedRepo & The GitHub generated id of the repository currently selected by the user\\ \hline
    GitLogin & The users GitHub login, used to bind commits to the user\\ \hline
    DateCreated & This field stores the date when the user was first created\\
    \hline
\end{tabularx}
\vspace{0.5cm}

\subsection*{Role}
The different roles that the users can be assigned to. These are hierarchy so that the highest gains access to the subset of this access level. \\

\vspace{0.5cm}
\begin{tabularx}{\linewidth}{| c | X |}
    \hline
    \rowcolor[gray]{0.8}
    \textbf{Field} & \textbf{Description} \\
    \hline
    Id & Unique id generated when the role is first created.\\ \hline
    Authority & This field contains the level of authority.\\
    \hline
\end{tabularx}
\vspace{0.5cm}

\subsection*{UserRole}
Both fields in this table are combined for the primary key such that each role can only be given to each user once and each users can attain several roles. \\

\vspace{0.5cm}
\begin{tabularx}{\linewidth}{| c | X |}
    \hline
    \rowcolor[gray]{0.8}
    \textbf{Field} & \textbf{Description} \\
    \hline
    User & Foreign key for the users id \\ \hline
    Role & Foreign key for the role id \\
    \hline
\end{tabularx}
\vspace{0.5cm}

\subsection*{Notification}
Notifications are used to prompt users about reflection sessions and such. \\

\vspace{0.5cm}
\begin{tabularx}{\linewidth}{| c | X |}
    \hline
    \rowcolor[gray]{0.8}
    \textbf{Field} & \textbf{Description} \\
    \hline
    Id & Unique id generated when the notification is first created\\ \hline
    Title & This field stores the title\\ \hline
   	Body & This field stores the entire message.\\ \hline
    DateCreated & Time stamp generated when the notification is first created.\\ \hline
    Unread & Boolean variable to state if the message is read.\\ \hline
    Cleared & Boolean variable to state if the message is cleared or not.\\ \hline
    NotificationType & The notification type, possible types are 'Reflection' or 'Other'.\\ \hline
    User & A foreign key to the user that received the notification.\\ 
    \hline
\end{tabularx}
\vspace{0.5cm}

\subsection{GitHub data}
In this section we discuss the storage of data collected from GitHub. When we started development we tested with the data stored at GitHub, so we asked GitHub for the data every time it was needed, this made the application slow so we decided to store GitHub data locally to enhance performance. 
The following data-types where stored in tables as described below.

\subsection*{Milestones}
Milestones are major goals consisting of several minor goals called issue, with or without a due date. \\

\vspace{0.5cm}
\begin{tabularx}{\linewidth}{| c | X |}
    \hline
    \rowcolor[gray]{0.8}
    \textbf{Field} & \textbf{Description} \\
    \hline
    GithubId & Unique id from github.com\\ \hline
    Name & The milestones name.\\ \hline
   	Description & A description of the milestone, like what features is to be implemented.\\ \hline
    Status & A variable to say if its open or closed.\\ \hline
    CreatedDate & The time stamp which the milestone is created.\\ \hline
    DueDate & A date which the milestone is due on. This is null if the milestone does not got a due date.\\ \hline
    ClosedDate & A time stamp when the milestone is closed.\\ 
    \hline
\end{tabularx}
\vspace{0.5cm}

\subsection*{Issues}
Issues are minor goals or bugs, these issues can be bound to a milestone or independent. \\

\vspace{0.5cm}
\begin{tabularx}{\linewidth}{| c | X |}
    \hline
    \rowcolor[gray]{0.8}
    \textbf{Field} & \textbf{Description} \\
    \hline
    GithubId & Unique id from github.com\\ \hline
    Name & The issues name.\\ \hline
   	Body & A description of the milestone, like what features is to be implemented.\\ \hline
    State & A variable to say if its open or closed.\\ \hline
    Number & A number unique to the repository which is used for referring the issue in the commit messages.\\ \hline
    Repository & A foreign key to the repository it is bound to.\\ \hline
    MilestoneNumber & The milestone which the issue is bound to.\\ \hline
    CreatedAt & The time stamp which the milestone is created.\\ \hline
    UpdatedAt & The time stamp which the milestone last was updated.\\ 
    \hline
\end{tabularx}
\vspace{0.5cm}

\subsection*{Commits}
At any given time a developer changes something in the source code it can be committed to the version control system, in this case git/github. These are the commits we store and use for reflection. \\

\vspace{0.5cm}
\begin{tabularx}{\linewidth}{| c | X |}
    \hline
    \rowcolor[gray]{0.8}
    \textbf{Field} & \textbf{Description} \\
    \hline
    GithubId & Unique id from github.com\\ \hline
    Message & The commit message which we gather tags from, these tags are marked hash tags.\\ \hline
   	Login & The GitHub user which did the commit.\\ \hline
    CreatedDate & The time stamp which the commit is created.\\ \hline
    Additions & A date which the milestone is due on. This is null if the milestone does not got a due date.\\ \hline
    Deletions & A time stamp when the milestone is closed.\\ \hline
    Total & A total of lines in the commit.\\
    \hline
\end{tabularx}
\vspace{0.5cm}

\subsection*{Repository}
The current data is stored about each repository, in the corresponding fields in the table bellow. \\

\vspace{0.5cm}
\begin{tabularx}{\linewidth}{| c | X |}
    \hline
    \rowcolor[gray]{0.8}
    \textbf{Field} & \textbf{Description} \\
    \hline
    GithubId & Unique id from github.com\\ \hline
    Name & The repository name.\\ \hline
   	Description & A description of the repository.\\ \hline
    CreatedAt & The time stamp which the commit is created.\\ \hline
    UpdatedAt & The time stamp recorded when the milestone last was updated.\\ 
    \hline
\end{tabularx}
\vspace{0.5cm}

\subsection{Collaboration data}
Teams are bound uniquely to a repository, each repository will only have one team bound to it and every user with access\footnote{Being a collaborator of the project on GitHub.} to the repository on GitHub will have the possibility to join the team.  


\subsection*{Team}
These teams of users are bound uniquely to a repository and there can only exists one team per repository. \\

\vspace{0.5cm}
\begin{tabularx}{\linewidth}{| c | X |}
    \hline
    \rowcolor[gray]{0.8}
    \textbf{Field} & \textbf{Description} \\
    \hline
    Id & Unique id generated when the team is created.\\ \hline
   	Owner & A foreign key to the user that created the team.\\ \hline
   	Name & The teams name.\\ \hline
    Repository & The githubId of the repository bound to the team.\\ 
    \hline
\end{tabularx}
\vspace{0.5cm}

\subsection*{TeamUser}
This contains the users attached to each team and data relevant. \\

\vspace{0.5cm}
\begin{tabularx}{\linewidth}{| c | X |}
    \hline
    \rowcolor[gray]{0.8}
    \textbf{Field} & \textbf{Description} \\
    \hline
    UserId & A foreign key to the users id.\\ \hline
   	TeamId & A foreign key to the teams id.\\ \hline
   	Role & The users role in the team. There are possible roles are 'Manager' and 'Developer'\\ \hline
    Active & A variable that states if the user has selected this team as his active. Only one can be active per user at any given time.\\ 
    \hline
\end{tabularx}
\vspace{0.5cm}

\subsection{Reflection data}
\subsection*{Notes}
These notes are bound to a team and there can only be created one for each team by a user each day. \\

\vspace{0.5cm}
\begin{tabularx}{\linewidth}{| c | X |}
    \hline
    \rowcolor[gray]{0.8}
    \textbf{Field} & \textbf{Description} \\
    \hline
    Id & Unique id generated when the note is created.\\ \hline
    Team & A foreign key to the team the note is bound to.\\ \hline
   	User & A foreign key to the user that created the note.\\ \hline
   	Mood & A level regarding the current mood the user that created the note recorded for that day. Mood is recorded as a number between 1-100 where 100 is very happy and 1 is very sad.\\ \hline
   	Contribution & The top 2 contributions done that day for the team by the user.\\ \hline
   	Improvements & The 2 things that can be improved by the user that day for the team.\\ \hline
   	Shared & Variable that states that the note is shared with the team.\\ \hline
    CreatedAt & The time stamp which the commit is created.\\ \hline
    UpdatedAt & The time stamp recorded when the milestone last was updated.\\ 
    \hline
\end{tabularx}
\vspace{0.5cm}

\subsection*{Workshop}
Managers or owners of teams can create reflection workshops, where it is generated a set of questions based on the tags used by the teams members in the selected period. \\

\vspace{0.5cm}
\begin{tabularx}{\linewidth}{| c | X |}
    \hline
    \rowcolor[gray]{0.8}
    \textbf{Field} & \textbf{Description} \\
    \hline
    Id & Unique id generated when the workshop is created.\\ \hline
   	TeamId & A foreign key to the team.\\ \hline
   	CreatedAt & The time stamp which the workshop is created.\\ \hline
   	DurationStart & The time stamp which the workshop period starts.\\ \hline
   	DateStart & The time stamp which the workshop period ends.\\ 
    \hline
\end{tabularx}
\vspace{0.5cm}

\subsection*{WorkshopQuestions}
This table holds the questions for each workshop and indicates if they are mandatory or not. \\

\vspace{0.5cm}
\begin{tabularx}{\linewidth}{| c | X |}
    \hline
    \rowcolor[gray]{0.8}
    \textbf{Field} & \textbf{Description} \\
    \hline
    Id & Unique id generated when the question is created.\\ \hline
   	Workshop & A foreign key to the workshop.\\ \hline
   	QuestionText & The generated question.\\ \hline
   	CommitTag & The tag which the question is generated from.\\ 
    \hline
\end{tabularx}
\vspace{0.5cm}