\chapter{Peaceful Banana Application}
\label{peacefulBananaApplication}
In this chapter we will introduce the PeacefulBanana tool, which is the prototype we developed to support this thesis. The focus will be to explain the concept of the tool and the features and functionality it provides. 
Peaceful Banana integrates with Github, collecting the most relevant data for reflection use. In order to describe the PeacefulBanana tool, an introduction of Github is required.\\

\section{Github}
\label{githubchapter}
GitHub is a web-based hosting service for software development projects that use the Git revision control system\cite{git,github}. GitHub offers both paid plans for private repositories, and free accounts for open source projects.

GitHub provides users with integrated issue tracking, code review, project wiki, useful statistics and more. 
Motivational points for choosing GitHub as the revision control system to integrate with are several. Some technical aspects are that GitHub provides developers with a simple to use API\cite{githubapi}, with libraries for integration with most of the commonly used programming languages, like Java\cite{egit}.\\
In addition to the technical aspects, GitHub is the most used revision control system as of May 2011, and is a tool many in our evaluation group already use in their development projects\cite{githubnumbers}.

\subsection{Repositories}
GitHub is a repository hosting service for software development projects. A repository contains all the project files, may it be code, images, and other documentation. This means that all content in a project is connected to this repository. In addition to documentation and file content, GitHub provides integrated issue tracking, which is connected to this repository and could also be connected to one or several milestones. 
\subsection{Commits}
Say some of your project files have been changed, for example some code snippet in a Java file. The way to save these changes to your local branch is by commits. A commit can consist of code line additions, deletions, files added, changed or removed. When you are ready to save the changes, you commit them in a commandline together with a commit message which is a short text describing the changes you have done. \\
When a user is ready to push one or more local commits to the project repository, it can be done via the command git push in the commandline. It is now the HEAD revision and the new code and commits can be seen on the github page.
\subsection{Milestones and Issues}
\begin{quote}
\em GitHub Issues can be assigned to a user to make it easy to know who's working on what, or which issues you need to tackle next.
\end{quote}
Every GitHub repository has an issue tracker, that allows users to track bugs and focus on features. Milestones and issues help manage large projects, where issues especially makes for a great TODO list, similar to a product backlog. Only {\bf collaborators} can create and view issues on private repositories. On public repositories anyone can create and view issues. 
\begin{itemize}
\item GitHub issues can be assigned to a user, which makes it easy to know who's working on what, or which issues should be handled next. Milestones are a good way of helping team members to work towards a goal. A team can set a due date, name a milestone and then start assigning issues to that milestone. An example of a Milestone could be a due date of a project demo or delivery. Any number of issues can then be assigned to this milestone, and thus be connected to it. 
\item Issues know all about commits. Github enables referencing and closing issues with Commit Messages. By using a few simple keywords you can close an issue right from a commit message, or just leave a note on the issue.The syntax to do this is as follows: To close issue \#35 , a commit message containing 'closes \#35' , will close issue number 35 when pushed to github. Other keywords are: close, closes, closed, fixes, fixed. \\
To leave a note on issues can be done by simply mentioning the issue number without any keywords in a commit message. F.ex "This commit references \#35". Anyone with write access to a repository may close an issue or leave a note.
\end{itemize}

\section{What is PeacefulBanana?}
The prototype will be a tool supporting reflection. The PeacefulBanana tool will integrate with Github and the features Github provides, see section \ref{githubchapter}. The tool will not provide the same possibilities already found on Github, but integrate with it using the existing data and provide an additional layer of features specific for aiding reflection in software development projects, for example prompting the user for their mood each day allows for a mood graph over time. PeacefulBanana is developed as a web-application using the Grails web framework and Java programming language. The tool will work in all modern web-browsers, on platforms like the PC, smartphones and tablets. 
The tool was developed with software development related artifacts collected from GitHub in mind. PeacefulBanana will be evaluated with software development teams that have adopted an agile process model. Most agile process models feature reflection sessions in some form, where this tool will be used to promote reflection. Although it is aimed at agile teams, any process model featuring some kind of reflection could benefit from using this tool in some way.  

\section{What does it do?}
The prototype is a tool that allow users to share and reflect upon their experiences in development projects. The tool integrates with GitHub and focus on collecting data from commits, milestones, issues as well as comments and references connected to these. The tool will scaffold these data and present them to the users, acting as input for triggering reflection. The reflection will then be captured and stored so the user can review these at a later date, or they can be shared and used in team reflection sessions. The tool is designed to support both individual and collaborative usag. It can be used in a scaffolded way or just browsed when the user wishes to. The tool provides several opportunities to the users:
\begin{description}

	\item{Provide a scaffolded overview of the project} 
	\begin{itemize}
	\item Users can choose what project to retrieve data from, see the projects milestones, issues and generate data relevant for reflection from these.
	\item The team can see when an issue has been closed, and what milestones they are connected to.
	\item The team can see which milestones were closed and when, and easily see if the team met its milestone deadlines. 
	\item The team can see which team members have done what.
	\end{itemize}

	\item {Reflection sessions} 
	\begin{itemize}
	\item What did the team do well in milestone x
	\item What could have been done differently when solving issue nr x
	\item Why didn't the team meet the deadline for milestone x
	\item Quick summaries of statistics, trending issues, tagclouds etc.
	\end{itemize}

	\item {Individual 5-minute daily reflection} 
	\begin{itemize}
	\item Why did I do things a certain way?
	\item What were my top 2 contributions today?
	\item Identify top two factors I can improve on?
	\end{itemize}
\end{description}

\section{Main goal}
Is to create a reflection tool for software development teams offering multiple representation of data. % Short story

\section{Functionality}

Reflection directed functionality:
\begin{itemize}
\item Collecting data - How a user can retrieve data and annotate it. 
\item Sharing data - How users can work together in groups to collaborate with data. 
\item Reflection triggers - Functionality developed specifically for helping users reflect on their work.
\end{itemize}

\subsection{Authentication}
The application will feature an independent authentication, with the possibility of getting access to data from Github by using OAUTH2 tokens. When the user first uses the tool, he will be asked to authenticate with Githubs authentication page asking the user to log in and authorize the PeacefulBanana tool. When this is done the tool receives a token it can use on behalf of the user to retrieve data from Github. The token is stored in a secure database and linked with the user logged in at the time. 

\subsection{Reflection triggers}
\paragraph{Scaffolded notes}\mbox{}\\
Tool should provide the ability to take notes but in a scaffolded way. It should not be possible to make too general notes. The tool will provide this through coupling with certain elements in the milestones. This will make notes connected to a certain commit or event in the project trajectory. 

\paragraph{Tagclouds}\mbox{}\\
Tool should provide the ability to create tagclouds based on data devided in time. So that you can see trending events, this will enable the user to see how their work is affecting the teams work overall. These tagclouds will allso give the team an indication on what the worked with over a period of time, and tags will be given a weight to show which of them that has been worked with the most. These weights will determine the words size based on ho

\paragraph{Sharing of data}\mbox{}\\
What data should be shared and how should the sharing be done in order to promote reflection? What functions should the tool provide to the user in addition to the github data that can be used for reflection?

\paragraph{Mood graphs}\mbox{}\\
A user can connect his mood to data. Mood data should be retrieved quite often(2-3 times each week) in order for the data to give something meaningfull back to the users.
User will have the option to share mood-data, creating a shared average mood of the project, a milestone or even on a single commit, giving an indication of how the work is perceived by the other group members. 
