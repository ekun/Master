\chapter{Peaceful Banana Application}
\label{peacefulBananaApplication}
In this chapter we will introduce the PeacefulBanana tool, which is the prototype we developed to support this thesis. The focus will be to explain the concept of the tool and the features and functionality it provides. 

\section{What is PeacefulBanana?}
Peaceful Banana is a tool aimed towards aiding reflection in teams. It integrates with GitHub, collects the most relevant project artifacts and scaffolds these in order to trigger and promote reflection in teams. GitHub is explained in section \ref{githubchapter}. The tool will not provide the same possibilities already found on Github, but integrate with it and provide an additional layer of features specific for aiding reflection in software development projects. The tool will collect different data relevant for reflection, for example prompting the user for their mood each day, which allows for a mood graph over time. PeacefulBanana is developed as a web-application using the Grails web framework and Java programming language. The tool will work in all modern web-browsers, on platforms like the PC, smartphones and tablets. 
The tool was developed with software development artifacts collected from GitHub in mind. PeacefulBanana will be evaluated with software development teams that have adopted an agile process model. Most agile process models feature reflection sessions in some form, where this tool will be used in order to promote and enhance reflection. Although it is aimed at agile teams, any process model featuring some kind of reflection session could benefit from using this tool in some way. Similarly the tool can also be used individually at any time. 

\section{What does it do?}
PeacefulBanana is a tool that allows users to share and reflect upon their experiences in development projects. The tool integrates with GitHub and focus on collecting data from commits, milestones, issues as well as comments and references connected to these. The tool will scaffold these data and present them to the users, acting as input for triggering reflection. The reflection will then be captured and stored so the user can review these at a later date, or they can also be shared and used in team reflection sessions. The tool is designed to support both individual and collaborative reflection. It can be used during the team reflection workshop or just browsed individually when the user wishes to. The tool provides several opportunities to the users:
\begin{description}

	\item{Provide a scaffolded overview of the project} 
	PeacefulBanana provides users and teams with an overview of the project:
		\begin{itemize}
		\item Users can choose what team-project to retrieve data from, see the projects milestones, issues and generate data relevant for reflection from these.
		\item The team can see when an issue has been closed, and what milestones they are connected to.
		\item The team can see which milestones were closed and when, and easily see if the team met its milestone deadlines. 
		\item The team can see which team members have done what.
		\end{itemize}

	\item {Individual 5-minute daily reflection} 
	Each day a notification will prompt the user to do a daily 5-minute reflection. This daily reflection presents the user with data for the last 24 hours, like commit activity and a tagcloud. This data provides the user with additional information and triggers reflection upon that days work. The daily summary collects input from the user:
		\begin{itemize}
		\item The users mood that particular day, ranging 5 steps from 0 (Very sad) to 100 (Very happy)
		\item The users top 2 contributions
		\item The users top 2 improvements
		\end{itemize}
	PeacefulBanana stores each of these daily summaries in a database. Each reflection summary can be shared with the users team if they choose to. The mood data from each day is used towards a collaborative mood-average graph, which can be used in the team reflection sessions described below

	\item {Reflection sessions} 
	The team, or the team leader can use a datepicker and create a workshop from a selected time and date. The workshop features some mandatory questions
	related to team work and reflection. Additionally the team can choose a set of tags to generate questions from. Quick summaries of project statistics, trending issues, tagclouds and more.The finished workshop template can be printed and
	handed out to the team at the workshop, providing quick summaries of project statistics, trending issues, tagclouds and questions that acts as reflection triggers\\
	Examples of such questions: 
		\begin{itemize}
			\item What were your initial expectations to this iteration? Did these expectations change during the iteration? How? Why?
			\item What could be done to improve team collaboration?
			\item Talk about any disappointments or successes of your project. What did you learn from it?
			\item You have had a high activity working with \#framework Did you experience any particular problems with this tag? Why or why not?
			\item What did you learn from working with the issue 'daily summary \& reflection notes' (\#17)?
			\item The team didn't meet the deadline for milestone \#20 - Midterm Report, did the team experience any particular problems?
		\end{itemize}
\end{description}

\section{Main goal}
Is to create a reflection tool for software development teams offering multiple representation of data. % Short story

\section{Functionality}

Reflection directed functionality:
\begin{itemize}
\item Collecting data - How a user can retrieve data and annotate it. 
\item Sharing data - How users can work together in groups to collaborate with data. 
\item Reflection triggers - Functionality developed specifically for helping users reflect on their work.
\end{itemize}

\subsection{Authentication}
The application will feature an independent authentication, with the possibility of getting access to data from Github by using OAUTH2 tokens. When the user first uses the tool, he will be asked to authenticate with Githubs authentication page asking the user to log in and authorize the PeacefulBanana tool. When this is done the tool receives a token it can use on behalf of the user to retrieve data from Github. The token is stored in a secure database and linked with the user logged in at the time. 

\subsection{Reflection triggers}
\paragraph{Scaffolded notes}\mbox{}\\
Tool should provide the ability to take notes but in a scaffolded way. It should not be possible to make too general notes. The tool will provide this through coupling with certain elements in the milestones. This will make notes connected to a certain commit or event in the project trajectory. 

\paragraph{Tagclouds}\mbox{}\\
Tool should provide the ability to create tagclouds based on data devided in time. So that you can see trending events, this will enable the user to see how their work is affecting the teams work overall. These tagclouds will allso give the team an indication on what the worked with over a period of time, and tags will be given a weight to show which of them that has been worked with the most. These weights will determine the words size based on ho

\paragraph{Sharing of data}\mbox{}\\
What data should be shared and how should the sharing be done in order to promote reflection? What functions should the tool provide to the user in addition to the github data that can be used for reflection?

\paragraph{Mood graphs}\mbox{}\\
A user can connect his mood to data. Mood data should be retrieved quite often(2-3 times each week) in order for the data to give something meaningfull back to the users.
User will have the option to share mood-data, creating a shared average mood of the project, a milestone or even on a single commit, giving an indication of how the work is perceived by the other group members. 
