\chapter{Problem Elaboration}
\label{problemelaboration}
In this chapter we will elaborate our problem by defining our task and presenting our high level requirements.
This chapter describes in more details the problem that you are addressing and that
has been briefly outlined in the Introduction. This chapter might contain scenarios that
illustrate the problem and help identifying requirements.

\section{Problem Definition}
\label{problemdefinition}

To answer our research questions we will develop a web-tool that allow users to create, share and reflect upon agile project experiences annotated with data relevant for reflection. 
The experiences will be presented in several different ways like graphs and tagclouds. The tool will be designed with reflection in mind, but the tool will also function as 
a tool that fits several scenarios and can be used by different groups of users. 

\section{Reflection and technology}
The world wide web and modern technologies provides easy access to enormous amounts of information. This means that in order to learn, learners must be able to make sense of the information collected.  
In order to achieve this and make conscious decisions of how to use information, learners need to reflect on the information they collect. Reflection upon the process of solving problems is necessary to achieve a good result and to improve the ability to learn from experiences. When supporting learning with technology, this technology should promote these aspects within learning\cite{Lin1999}

The main goal is to provide technology that enables efficient information retrieval, and to provide scaffolds that support reflective thinking and problem solving. In our task this means utilizing collaborative learning experiences. 

\section{Scenarios}
\label{problemdefinition}
Based on related work, we have developed three scenarios to show what we want our tool to support in terms of collaborative reflection and learning. These three scenarios will provide a basis for the requirement elaboration and design choices in development of the tool. 
\\
At the Norwegian University of Science and Technology, NTNU, a group of students are taking the course IT2901 - Informatics Project II\footnotemark.
In the course, the focus is not only on the project itself, but also on the process the group steps through in order to reach their goal, including group roles, distribution of work etc. At the end of the course, students have to present a product report, with their project results, and a process report, with their project process results and experiences. In addition to this, groups participate in a retrospective workshop at the end of the course. In these three scenarios, our users are students and part of a group in the course , with age ranging from 22 and up to 28 years old.
\\
The group is going to work on a customer-driven project, using an agile methodic like scrum. The group will be using Github as a versioning-control system. During the project work the team will use Github to collect information. The PeacefulBanana tool can then be used to gather this information and present it in a scaffolded way mainly to be used in reflection sessions. Users can though, at any time go into the application and gather relevant data if they want to. This data will help the group to see trending issues and problems they have come upon in the process. Each member of the group will register as a user on the PeacefulBanana tool. The users will then be connected together as a group. 
\\
The two scenarios described in the next sections provides examples of how we envision the usage of the PeacefulBanana tool in three different settings. First as a individual tool on a daily basis, then as part of a team collaborative reflection session.
These two scenarios were developed early in the development process. The first scenario features using the tool at the end of each working day as inspired by [ref to the 5 minute daily reflection session]. The second scenario where in the end of each iteration there is a retrospective reflection session. 
\\

\footnotetext{Informatics Project II is a course where students from the Informatics education program is put together in groups working on different software development projects. Key objectives of this course are gaining practical experience in GROUP-ORIENTED software engineering for a customer, covering the whole life-cycle of a software project - \url{http://www.ntnu.edu/studies/courses/IT2901}}

\subsection{Scenario 1 - Individual use on a daily basis}
\label{scenario1}
In this scenario, our students will be using the tool on a daily basis at the end of each working day. When the students enter the site, they will get a notification with a request to do the daily status update. After clicking the notification, the student is presented with a summary of their individual activity in the last 24 hours. The user are then prompted to input todays mood, their top two contributions \emph{"What did I do good?"} , and their top 2 points to improve on \emph{"What could I do better?"}. Finally the user can choose to share these experiences with the team for collaborative use, and then submit the form. The daily status updates are saved and can be reviewed at any time, and also shared any time by the student. 

\subsection{Scenario 2 - Team use after each iteration}
\label{scenario2}
In this scenario, the students will be using the tool as part of the agile methodic two-week reflection sessions. Students will in this scenario use the tool to indicate how the project has been progressing over the last iteration, tightly coupled with one or several milestones. The students will generate tagclouds based on the trending issues in the relevant milestones, activity graphs and mood trajectories. Also if the users have chosen to share any of the individual reflection notes from scenario 1, these will be visible and can be used to draw conclusions from the previous iteration, and make comparisons. The tool will enable a group to see the whole iteration more clearly, but also be able to dive into certain issues or milestones that showed to be of particular interest, and thus create a discussion around the experiences made by the team members.  