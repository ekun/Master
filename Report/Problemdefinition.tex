\chapter{Problem Elaboration}
\label{problemelaboration}
In this chapter we will elaborate our problem by defining our task and presenting our high level requirements.
This chapter describes in more details the problem that you are addressing and that
has been briefly outlined in the Introduction. This chapter might contain scenarios that
illustrate the problem and help identifying requirements.

\section{Problem Definition}
\label{problemdefinition}

To answer our research questions we will develop a web-tool that allow users to create, share and reflect upon agile project experiences annotated with data relevant for reflection. 
The experiences will be presented in several different ways like graphs and tagclouds. The tool will be designed with reflection in mind, but the tool will also function as 
a tool that fits several scenarios and can be used by different groups of users.  

\section{Scenarios}
\label{problemdefinition}

\subsection{Scenario 1}
\label{scenario1}
Using the tool as part of the agile methodic ‘Scrums’ retrospective-reflection meetings reflecting upon the last week/2weeks or what the iteration duration might be.  The tool will provide users with the option of generating a report/graphs/other data that can be used to remember certain situations and how the group/persons dealt with this matter. The team will be able to generate tagclouds from commit messages during an iteration to see what issues or ‘tags’ were the most common for that period. This enables the team to reflect on the different issues. 
\subsection{Scenario 2}
\label{scenario1}
Using the tool as part of the course final workshop, reflecting on the project process as a whole. 
This scenario uses the tool to indicate how the project has been long-term. What issues are common over several milestones(tagclouds), graphs etc. Also possibly mood-trajectories both personal and as a group. The tool will enable a group to see the whole project course more clearly, but also be able to dive into certain issues or milestones that showed to be of particular interest, which in turn can aid reflection. 
\subsection{Scenario 3}
\label{scenario1}
for use for later reflection: Using the tool daily as a personal tool making relevant notes and other relevant things like feelings in order to reflect upon this later. Project members use github daily for commits, comments and as a general version-control and issue-tracker tool. By integrating the github data with our tool, project members could see commits made, milestones reached etc. and take notes regarding these events. Other functions is adding relevant tags, emotions and such. Using the tool daily means all the notes, emotions etc. are fresh and correct. Later in the retrospectives mentioned in scenario 1 and 2 , members can look back and remember emotions and what actually went on. This leads to better reflection, as it has been shown that memories change over time, so inputting the fresh experiences for later use is important. 
