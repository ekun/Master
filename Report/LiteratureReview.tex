%!TEX root = main.tex
\section{Literature Review}
In this chapter we will elaborate on the theory on reflection and how technology fits into this domain. 

\subsection{Reflecting on reflection}
\subsection*{Important aspects}
D.Talby \cite{Talby2006} focuses on four aspects proven to be important for reflection. They also proposed a technique for reflection, where the teamleader chose the subject beforehand. This has the advantage of not spending time to decide on a subject during the meeting itself, which in turn reduces the time needed. Also there was a reduced risk of selecting 'wrong' subjects, since the moderator(teamleader) was a part of the development team and were aware of the day-to-day problems.\\
Further Talby found that reflection subjects needed to be:
\begin{itemize}
\item Relevant to the team in its entirety, and not personal quarrels
\item Organizational issues
\item Issues there are disagreement on, but not technical problems
\end{itemize}
Further, open discussions are sufficient to achieve bottom-line results, as opposed to more structured reflection sessions.\\
A written summary of the reflection session was seen as highly important, even only in an informal way, such as email. This summary was important in order to exclude any later differencies on what was agreed upon and in what way.
\subsection*{Reflection levels}
There are many ways of using reflection as a concept. Fleck and Fitzpatrick 


Reflection is a wide concept, and it is being used in diverse ways. In [19]
Fleck and Fitzpatrick synthesize the literature looking on aspects such as
purposes of reflection, conditions for reflection and levels of reflection (where
the levels capture the behaviors and activities associated with reflection).
The authors points out that “[. . . ] the interest in reflection and technologies
to support reflection has expanded beyond these more traditional domains
to a range of new areas, with reflection as a topic in its own right”. They
say that reflection is a time consuming process, that requires the right environment
and encouragement to happen. The authors identifies five levels of
reflection that spans from the lowest level where the user merely describes the
situation to the highest level, where the learner is “taking into consideration
aspects beyond the immediate context, for example moral and ethical issues,
and wider sociohistorical and politico-cultural contexts”. The techniques for
supporting reflection on the different levels identified in [19] will be used as
a background when designing the Timeline application.

% These need to be organized into topics relevant for our thesis instead of just listing the reviews
\subsection{Tag Clouds: Data Analysis Tool or Social Signaller?}\cite{Hearst2008}
Marti A. Hearst and Daniela Rosner examine the recent information visualization phenomenon known as tag clouds, which are an interesting combination of data visualization, web design element, and social marker. Using qualitative methods, they found evidence that those who use tag clouds do so primarily because they are perceived as having an inherently social or personal component, in that they suggest what a person or a group of people is doing or is interested in, and to some degree how that changes over time. The primary reasons people object to tag clouds are their visual aesthetics, their questionable usability, their popularity among certain design circles, and what is perceived as a bias towards popular ideas and the downgrading of alternative views.

\subsection{Shared timeline and individual experience: Supporting retrospective reflection in student software engineering teams}\cite{Krogstie2009}
Birgit R. Krogstie and Monica Divitini To help SE student teams learn from their project process, we propose the use of facilitated postmortem workshops in which each team reconstructs its project timeline. Individual team members’ experience of the project is included by team members drawing individual ‘experience curves’ along the timeline. The approach is based on current industry practice and adapted in accordance with theory on reflection and learning.

\subsection{The functions of multiple representations}\cite{Ainsworth1999}
Shaaron Ainsworth: Multiple representations and multi-media can support learning in many different ways. In this paper, it is claimed that by identifying the functions that they can serve, many of the conflicting findings arising out of the existing evaluations of multi-representational learning environments can be explained. This will lead to more systematic design principles. To this end, this paper describes a functional taxonomy of MERs. This taxonomy is used to ask how translation across representations should be supported to maximise learning outcomes and what information should be gathered from empirical evaluation in order to determine the effectiveness of multi-representational learning environments.

\subsection{Evidence-Based Timelines for Agile Project Retrospectives – A Method Proposal}\cite{Bjarnason2012}
In this paper we look at the findings of Elizabeth Bjarnason and Björn Regnell about retrospective analysis of agile prosjects. Whether it can support identification of issues through team reflection and may enable learning and process improvements. Basing retrospectives primarily on experiences poses a risk of memory bias as people tend to remember events differently which again can lead to incorrect conclusions. This bias is enhanced when looking over a longer periode compared to interation retrospectives. To support teams getting a joint view of projects the article suggests creating a method for visualizing an evidence-based project timeline by illustrating aspects such as business priority, iterations and test activities. The method complements the already existing experience-based approach by providing objective data as a starting point for reflection.