\section{Literature Review}
Review of different literature relevant to the thesis.

\subsection{Tag Clouds: Data Analysis Tool or Social Signaller?}\cite{Hearst2008}
Marti A. Hearst and Daniela Rosner examine the recent information visualization phenomenon known as tag clouds, which are an interesting combination of data visualization, web design element, and social marker. Using qualitative methods, they found evidence that those who use tag clouds do so primarily because they are perceived as having an inherently social or personal component, in that they suggest what a person or a group of people is doing or is interested in, and to some degree how that changes over time. The primary reasons people object to tag clouds are their visual aesthetics, their questionable usability, their popularity among certain design circles, and what is perceived as a bias towards popular ideas and the downgrading of alternative views.

\subsection{Shared timeline and individual experience: Supporting retrospective reflection in student software engineering teams}\cite{Krogstie2009}
Birgit R. Krogstie and Monica Divitini To help SE student teams learn from their project process, we propose the use of facilitated postmortem workshops in which each team reconstructs its project timeline. Individual team members’ experience of the project is included by team members drawing individual ‘experience curves’ along the timeline. The approach is based on current industry practice and adapted in accordance with theory on reflection and learning.

\subsection{The functions of multiple representations}\cite{Ainsworth1999}
Shaaron Ainsworth: Multiple representations and multi-media can support learning in many different ways. In this paper, it is claimed that by identifying the functions that they can serve, many of the conflicting findings arising out of the existing evaluations of multi-representational learning environments can be explained. This will lead to more systematic design principles. To this end, this paper describes a functional taxonomy of MERs. This taxonomy is used to ask how translation across representations should be supported to maximise learning outcomes and what information should be gathered from empirical evaluation in order to determine the effectiveness of multi-representational learning environments.

\subsection{Evidence-Based Timelines for Agile Project Retrospectives – A Method Proposal}\cite{Bjarnason2012}
In this paper we look at the findings of Elizabeth Bjarnason and Björn Regnell about retrospective analysis of agile prosjects. Whether it can support identification of issues through team reflection and may enable learning and process improvements. Basing retrospectives primarily on experiences poses a risk of memory bias as people tend to remember events differently which again can lead to incorrect conclusions. This bias is enhanced when looking over a longer periode compared to interation retrospectives. To support teams getting a joint view of projects the article suggests creating a method for visualizing an evidence-based project timeline by illustrating aspects such as business priority, iterations and test activities. The method complements the already existing experience-based approach by providing objective data as a starting point for reflection.