%!TEX root = main.tex
\chapter{State of the art}
\chapter{Literature Review}
Review of different material

\section{Tag Clouds: Data Analysis Tool or Social Signaller?}\cite{Hearst2008}
Marti A. Hearst and Daniela Rosner examine the recent information visualization phenomenon known as tag clouds, which are an interesting combination of data visualization, web design element, and social marker. Using qualitative methods, they found evidence that those who use tag clouds do so primarily because they are perceived as having an inherently social or personal component, in that they suggest what a person or a group of people is doing or is interested in, and to some degree how that changes over time. The primary reasons people object to tag clouds are their visual aesthetics, their questionable usability, their popularity among certain design circles, and what is perceived as a bias towards popular ideas and the downgrading of alternative views.

\section{Shared timeline and individual experience: Supporting retrospective reflection in student software engineering teams}\cite{Krogstie2009}
Birgit R. Krogstie and Monica Divitini To help SE student teams learn from their project process, we propose the use of facilitated postmortem workshops in which each team reconstructs its project timeline. Individual team members’ experience of the project is included by team members drawing individual ‘experience curves’ along the timeline. The approach is based on current industry practice and adapted in accordance with theory on reflection and learning.

\section{The functions of multiple representations}\cite{Ainsworth1999}
Shaaron Ainsworth: Multiple representations and multi-media can support learning in many different ways. In this paper, it is claimed that by identifying the functions that they can serve, many of the conflicting findings arising out of the existing evaluations of multi-representational learning environments can be explained. This will lead to more systematic design principles. To this end, this paper describes a functional taxonomy of MERs. This taxonomy is used to ask how translation across representations should be supported to maximise learning outcomes and what information should be gathered from empirical evaluation in order to determine the effectiveness of multi-representational learning environments.

This chapter provides an overview of the literature. It positions your work with respect to
work already done by others.

Important points to remember in the literature review:
\begin{itemize}
\item be organized around and related directly to the thesis or research question you are developing
\item synthesize results into a summary of what is and is not known
\item identify areas of controversy in the literature
\item formulate questions that need further research
\end{itemize}

\section{Releated Work}
Is there any other similar tools already developed? There are project development tools which integrates with Github in order to populate and enhance Kanban boards, but these have no features for experience collection and promoting reflection. 
Tools using github api: \\
\subsection*{HuBoard}
GitHub issues made awesome: \url{http://huboard.com/} \\
HuBoard is a project management tool, which takes your GitHub issues and milestones and generates a Kanban\footnote{Kanban is a software development method with focus on Just-in-time delivery. Developers pick tasks from a queue. \url{http://www.kanbanblog.com/explained/index.html}} board built from the GitHub api. \\
This tool features the idea of scaffolding issues and milestones into something useful for a different context, although it is limited to issues and have no features for capturing experience and promoting reflection. 
\subsection*{Agile Bench}
Agile Bench integrates with Github so programmers do not have to duplicate comments between two systems\footnote{Agile Bench Github integration: \url{http://support.agilebench.com/entries/21307153-GitHub-Integration}}. Instead they can make a comment via Github(the codebase) commits and this will push comments into the project management system. AgileBench supports these commit formats:
\begin{itemize}
\item \verb|[#story_id]| comment - i.e. "[\#5] Added documentation" - will add "Added documentation" to story \#5
\item \verb|[#story_id #story_id]| comment. i.e. "[\#5 \#6] Added even more documentation" - will add "Added documentation" to story \#5 and \#6
\item \verb|[Workflow State #story_id #story_id]| comment. i.e. "[In Progress \#5 \#6] Added even more documentation" - will add "Added even more documentation" to story \#5 and \#6 and move stories \#5 \& \#6 into the In Progress workflow state.
\end{itemize}
Agile Bench do not facilitate for experience collection in order to promote reflection.
\subsection*{AgileZen}
AgileZen also organizes work in a Kanban board, with user-stories\footnote{AgileZen: \url{http://www.agilezen.com}}. Zen features teams with users and different user-roles. It also allows artifacts like screenshots or specification documents to be attached to a story. It integrates with GitHub as a service hook\footnote{\url{https://help.github.com/articles/post-receive-hooks}}. In order for Zen to include your changesets from github, you need to refer to the story ID in your commits using the format \#123 or zen-123. This further builds on the idea of using \#hashtags to \textit{tag} relevant messages and identify them in your commits. 

%\subsection{Some other reflection stuff}
%To do
