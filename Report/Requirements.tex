%!TEX root = main.tex
\chapter{Requirements}
\label{chap:requirements}
\section{Functional requirements}
This is the list of functional requirements.
Here are some user stories:

\subsection{User stories}
\paragraph{Repositories}\mbox{}\\
\vspace{0.5cm}
 \begin{tabularx}{\linewidth}{| l | X | X |}
    \hline
    \rowcolor[gray]{0.8}
    \textbf{ID} & \textbf{As a User I want to..} & \textbf{so that..} \\
    \hline
    1. & get an overview of my github repositories & I can access them to view relevant data\\
    2. & see which repositories are most active  & I know when updates have occurred\\
    3. & generate a summary of issues worked with the last milestone & I can remember certain situations and reflect upon them\\
    4. & generate a summary of issues worked with during the whole duration of the project & I can reflect on how the process in a long-term scope\\
    \hline
\end{tabularx}
\vspace{0.5cm}

\paragraph{Commits}\mbox{}\\
\vspace{0.5cm}
 \begin{tabularx}{\linewidth}{| l | X | X |}
    \hline
    \rowcolor[gray]{0.8}
    \textbf{ID} & \textbf{I want to..} & \textbf{so that..} \\
    \hline
    1. & see what commits has been made in a repository. & I can see what has happened in the code base\\
    2. & generate tag clouds showing the most used tags in repository commits (over a period of time) & I can see trending issues/emotions for reflection use.\\
    3. & see what commits that has been done last sprint and when(in time) & I can see when the team was most efficient.\\
    4. & see how I managed my time & I can spend my time more efficiently\\
    5. & make short notes on a commit & I can revisit the commit in reflection sessions\\
    6. & add/connect feelings to a commit & I can revisit the commit in reflection sessions\\
    7. & add tags to a commit & I can revisit this and similar commits with the same tags\\
    8. & share my notes/feelings and make them available to the group & I can share my thoughts with the team for use in the reflection sessions\\
    \hline
\end{tabularx}
\vspace{0.5cm}

\paragraph{Milestones}\mbox{}\\
\vspace{0.5cm}
 \begin{tabularx}{\linewidth}{| l | X | X |}
    \hline
    \rowcolor[gray]{0.8}
    \textbf{ID} & \textbf{I want to..} & \textbf{so that..} \\
    \hline
    1. & see what Milestones that is being worked on & I can see if there is any unresolved issues attached to these\\
    2. & see what issues I have closed & I can reflect on how I solved it\\
    3. & see what issues are assigned to me & I can see what issues I need to look into\\
    4. & see what issues are newly created since last synchronization & I know if some need my attention\\
    5. & see what issues are common over several milestones & I can look into trending problems for use with reflection\\
    \hline
\end{tabularx}
\vspace{0.5cm}

\paragraph{My profile}\mbox{}\\
\vspace{0.5cm}
 \begin{tabularx}{\linewidth}{| l | X | X |}
    \hline
    \rowcolor[gray]{0.8}
    \textbf{ID} & \textbf{I want to..} & \textbf{so that..} \\
    \hline
    1. & link my PeacefulBanana profile with my GitHub account & I can retrieve data from GitHub\\
    2. & change my preferences  & they are correct\\
    \hline
\end{tabularx}
\vspace{0.5cm}

\section{Non-functional requirements}
This is the list of non-functional requirements.
