%!TEX root = main.tex
% Legger alt av usability stuff her enn så lenge

% Bruk testplan.docx og test-report.docx som en slags mal til hva som skal være med i plan og rapport. Legg det inn her 
% Se på Usability Master 2 Svanaes.pdf for et forslag til oppsett av dette. F.eks en section med gjennomføring og en med resultater. Slå sammen stuffet fra testplan.docx og test-report.docx til en fin ting!  
\chapter{Usability}
% Her bør vel det nevnes det fra testplanen, kjapt hva vi ville og resultatene stå. Vi bør også forklare testplanen mer et annet sted, såvidt i research method? Men kanskje også reintrodusere test-kapitlet og legge alt fra intro til resultater der? Bør vel ikke stå altfor mye i intro kapitlet dog
% Usability testen er jo ikke den primære evalueringen, men iom. at vi har begrenset med data bør vi nok fokusere litt på den. 
In this section we will describe the results and observations gained through usability tests. Before we conducted the usability test we created a usability test plan: Appendix \ref{chap:usability}, where all the different parts of the usability test is described in detail.
\section{Test Goals}
Performing usability tests on the PeacefulBanana prototype serves primarily two goals:
\begin{itemize}
\item Identify possible problems or breakdowns in the design\cite{ref:30} early on in the design process.
\item Identify the relationship users have with the aspect of reflection and sharing personal experiences.
\end{itemize}
Typical problems identified would be text representations or the placement of design elements, that are not intuitive for the user during use. It would be a concern if the user can't figure out how to use certain features of the application. Identifying these problems as early as possible will lead to a better end design. \\
Secondly an objective of the usability test was to identify how users act and think about their daily experiences, how they react to the notion of reflecting on them and if sharing their private thoughts is a problem. 

\section{Pilot Test}
After finalizing the usability test plan: Appendix \ref{chap:usability}, a pilot test was conducted prior to the usability-test[ref 22]. The pilot test allows for an evaluation of the test plan itself and the questionaires before doing the actual usability test. This means the pilot test is a "test of the test", where the goal is to evaluate and verify that the test itself is well-formulated. We chose a fellow student as our pilot-tester, in order to check whether the test script was clear, that the tasks were appropriately difficult, and that the data collected can be meaningfully analyzed. 
It also allows the "tester" to practice the execution and guidance, before actually performing the tests. \\
In the pilot test for PeacefulBanana, all of the aspects above were evaluated and a few tweaks were made to the tests, making it more streamlined. Also a few, smaller bugs in the application were discovered and fixed. The test introduction was rewritten, since the pilot-tester showed some confusion in a few of the tasks. 

\section{Usability Test Results}
The findings acted as valuable feedback to our delivery cycle, and were used for improving the design of the application. 

% Usability test results: Skrive hvordan de svarte på spørsmålene i test-planen, hvordan det gikk. Evt problemer identifisert 
% Questionaire resultater: Demografi info, diagrammer etc. Vi har jo ikke nok data egentlig, men vi får lage så vi har nok data tbh. 
% System feedback: Noe om systemet i seg selv.. hvordan det er å bruke, design etc.
% Refleksjonsfeedback: Hvordan var det å skrive ned refleksjonene sine om erfaringene, og hvordan er det å dele de med andre? Mer diagrammer hvis nødvendig. 
