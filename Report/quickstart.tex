\chapter{PeacefulBanana Quick Start}

\begin{figure}[h!]
\label{logo}
\centering
	\includegraphics[width=\textwidth]{logo}
\caption{The user has clicked the notification icon and can see his new notifications}
\end{figure}

This chapter features the PeacefulBanana quick start guide, given to the students for our evaluation. \\
Peaceful Banana is a tool aimed towards aiding reflection in teams. It integrates with GitHub, collects the most relevant project artifacts and scaffolds these in order to trigger and promote reflection in your team.
See you and your co-workers latest activity and use tag-cloud or statistics to create your daily reflection notes. PeacefulBanana allows you to choose what to share, and what to keep private! Reflect on your individual work by revisiting reflection notes, or reflect on your team's activity through reflection workshops. Much more inside!

% Denne delen bør ikke være for lang, men vise kort med screenshots hva som skjer hvor og mulighetene. Denne bør gjøres snarest og minst før evalueringen. 
% Se evt timeline mastern, der har de noe lignende som kan brukes som inspirasjon =)

\section{Getting started}
In order to use the Peaceful Banana application you need access to a device with an up-to-date internet browser, such as Firefox, Google Chrome, Safari, Opera or Internet Explorer(version 7 and up). Other browsers present on tablets and smartphones may also work, but we suggest using one of the mentioned ones. 

\subsection{Registration}
Registration process and confirmation of email. 
\subsection{Choosing team}
Create or choose a team. How to change a team. + rest of team tab. 
\subsection{Repository tab}
Repository tab explanation - milestones, issues, tagcloud etc.
\subsection{Reflection tab}
Reflection tab, notes, mood graph, workshop preparation.
\subsection{Workshop tab}
Workshop tab and how to create a workshop etc. 


