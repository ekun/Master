%!TEX root = main.tex
% CONCLUSION
\chapter{Conclusion}
 
\section{Summary}
In this theses we have investigated how scaffolding of reflection triggers can enhance the reflection level with project artifacts collected from version control systems. We have developed a platform-independent prototype in order to see how to visualize and present data collected from the artifacts.

During the last decade there has been a huge focus on improving development methodology and the now renowned industry-standard is agile development methodology. Agile software development projects generate a lot of data per developer in the version control system and other tools relevant for project management, each of the commit to the version control system contains tiny notes relevant to the work they have done. Our investigation shows that there is little to no other projects done based on this data and this was backed up during our expert evaluation, the scientist pointed out that during all his time around agile development projects he had never see any tools like the one developed by us. Even though that there has been a great deal of research regarding reflection in development teams. When we designed the prototype we set up a set of requirements which where prioritized based on difficulty and time to completion.

The implementation was a result of the theoretical background and related work, evaluation and discussion with an expert in the area of agile development. Then the resulting prototype was evaluated with a focus group consisting of a agile development team.

We then analyzed the results together with the theoretical background and notes from the expert review, where we discovered that the tools features was found useful. However we also discovered that some of the key features could be improved with the notes gathered from both the expert review and the focus group interview.

\section{Discussion on our own work}
%%Discussion, with a critical reflection on the work that has been presented,
%pointing out strengths and weaknesses; e.g. how well have the research questions been answered, limitations in the research methods, …

The primary work done for this thesis was the development of the PeacefulBanana application. Developing PeacefulBanana challenged us to learn about developing applications for the web, and a large variety of different devices and platforms. We also got to explore web frameworks for building web-applications, specifically the Grails framework and the domain of responsive web design. In addition to working with new and upcoming technology like HTML5 and Twitter Bootstrap, we learned how to setup our own Ubuntu server, with Apache, Apache TomCat and SQL, all of which is highly relevant in the daily work of developers.\\
Working together on this thesis and application has been both fun and exciting, and has also given us valuable experience and knowledge we will benefit from in our future work as software developers.

During development, several changes and plugins were made available for the Grails framework, some of which could have had a positive impact on the application and how it works. Because of time limitations, it was not plausible to implement these so late in the process, but it is a pity the application could not benefit from some of these additions. 

We first created our scenarios with evaluating the tool on software development teams in the IT2901 course at NTNU. Unfortunately the groups were very busy, and was not able to use the tool enough for us to conclude on the data gathered. Therefore we conducted a focus group with such a development team, where we gained valuable information on why they didn't use the tool as often, and also how they felt about the tool \emph{if they had used it}. As it turned out, our two scenarios functioned well in this setting, although they were first intended for evaluation over a longer period of time. The main reason for the group not using the tool, was simply that they were very busy and adopted the tool at a late stage in the project course. Feedback from the focus group was that if introduced with the tool from the beginning it would have been easier to remember to use it, as it would be part of their routine. Having the teams evaluate the application in a real-work setting over a longer time-period may have given us more results. 

Although usability was not our top-priority when developing the application, the usability test with real life users showed that usability has a large impact. Although we gained a lot of valuable feedback from our usability test, we think that an even earlier usability test could have provided us more insight on how to make the application more attractive and usable for real-life users. 

When we look back at the work we have done in this thesis, we are both happy and satisfied with what we achieved. We got to work with new technology and create something we never had worked with before. We got to dive into the inner workings of the most popular version-control system \emph{GitHub}, and take part of the whole development process from initial ideas to mock-ups, development, testing, deployment and evaluation of the final product. All in all the application developed for this thesis is something we look back on with pride and satisfaction. 

\section{Future Work}
As the interview with the focus group discovered there were several minor improvements both in how certain functions are implemented, but also in how the feature itself is designed visually. 

Technically we would like the tool to have a tighter integration with the version-control-system so that the synchronization itself will go automatically.

As the focus group helped us discover that the daily summary should hold more information, cause you can not really derive what you have done the last 24 hours, and the feature it self could be improved so that the user can edit the note through out the day. However we feel that this could interfere with the reflection level you would achieve with the solution we have created since noting a lot as things occur is not reflecting. They also felt like there was missing a field which they could record completely what they had done that day. It is clear that the daily summary needs to be improved, the fact that it does not provide the user with what they feel is enough data. We feel that an improvement of the current solution and the solution described above. By letting the user note what they have done continuously and answer the reflection specific questions only at the end of the day, the user could note all day long and reconstruct their day more easily and still reflect on a daily basis.

Regarding tag clouds it was discovered that it was hard to find the same tags in the team vs my tag cloud, because the tags is placed randomly and with a random color in both clouds. So we would place tags in the same region and with the same color so it would be  easier to see the relationship between the tag clouds.

