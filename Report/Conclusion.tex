%!TEX root = main.tex
% CONCLUSION
\chapter{Conclusion}

\section{Summary}
Summary of results. At this point results should be discussed also in
relation to research questions.

\section{Discussion on our own work}
Discussion, with a critical reflection on the work that has been presented,
pointing out strengths and weaknesses; e.g. how well have the research questions been answered, limitations in the research methods, …

\section{Future Work}
As the interview with the focus group discovered there were several minor improvements both in how certain functions are implemented, but also in how the feature itself is designed visually. 

Technically we would like the tool to have a tighter integration with the version-control-system so that the synchronization itself will go automatically.

As the focus group helped us discover that the daily summary should hold more information, cause you can not really derive what you have done the last 24 hours, and the feature it self could be improved so that the user can edit the note through out the day. However we feel that this could interfere with the reflection level you would achieve with the solution we have created since noting a lot as things occur is not reflecting. They also felt like there was missing a field which they could record completely what they had done that day. It is clear that the daily summary needs to be improved, the fact that it does not provide the user with what they feel is enough data. We feel that an improvement of the current solution and the solution described above. By letting the user note what they have done continuously and answer the reflection specific questions only at the end of the day, the user could note all day long and reconstruct their day more easily and still reflect on a daily basis.

Regarding tag clouds it was discovered that it was hard to find the same tags in the team vs my tag cloud, because the tags is placed randomly and with a random color in both clouds. So we would place tags in the same region and with the same color so it would be  easier to see the relationship between the tag clouds.

