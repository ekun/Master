\section{Research Method}
Table 1 \cite{Esearch2004} will be the basis of this design-science research and represents the following categories evaluation criteria’s will be constructed as described in \cite{Esearch2004}.
\subsection{Approach}
The approach will be the result of a design-oriented project, where the result will be evaluated based on the research questions defined above.
\subsection{Design as an artifact}
We will create a proof of concept prototype tool for supporting multiple representation for reflection in collaborative sessions. From this point the artifact will be referred to as the application.
\subsection{Problem relevance}
We want to explore the use of multiple representations to support collaborative reflection sessions. In order to learn from previous experiences we want to display experiences in ways that enhances the users learning experience. It shall be easy and simple to change between visualizations based on context and preferences. 

For example it shouldn’t be an alternative to watch experiences in a view that will list them at the same time or place.

It should also be listed notes from relevant experiences based on tags related to the experience being reviewed. \cite{Hassan-montero2006}
\subsection{Evaluation}
Table 2 in \cite{Esearch2004} will be used as a basis for evaluating and notes on how to evaluate each point are listed below in this subsection.
\subsubsection{Observational}
** Testcases **

After completing the test period all test-subjects will answer a form with a set of questions to evaluate the application and attend a short interviewed to get a better understanding of why the test-subject answered as he/she did and any other comments would be greatly appreciated.
\subsubsection{Analytical}
We will perform a dynamic analysis on the data collected from server and user feedback. During use the application will provide logs on how the users use the application and we will be able to draw simple usage-patterns from this. This will also help us in the analysis of the applications performance as well as if it is a useful tool.
\subsubsection{Experimental}
During development and testing we will be running simulations to ensure that the user experience is as expected. Case study and expert review will be with both artificial data and real experiences contributed by these groups, the artificial data will be used as basis for comparison during.
\subsubsection{Testing}
White Box testing will be executed during development of the application with artificial data to ensure that the function is working properly and as specified.
\subsubsection{Descriptive}
Here we will discuss a scenario where this application will demonstrate its full potential.

On a conference that goes over a weekend with multiple auditoriums there might be multiple sessions overlapping and this makes you pick the sessions you want to attend, by using a collaborative tool like the one we are creating users can contribute with notes from each of the sessions and link them to a place, time and generate tags on the session topic. 

This will then enable users to go back in time after the conference has ended and to relive some of the sessions through another users note. For instance you can go back to a specific stage and review all the notes created in that auditorium spread over a timeline. Or the user can search for experiences by using keyword \cite{Hassan-montero2006}, these keywords can be linked to a set of experiences whit other keywords the experiences got in common and will create a tag-cloud for each set of experiences.
\subsection{Research Contributions}
The main goal is to create a proof of concept application that will help users reflect on their past experiences in a collaborative session by using multiple representations as a tool. 
We will analyse the users experience in order to draw a conclusion on if multiple representations might help the users to reflect on their past experiences in a collaborative session.
\subsection{Research Rigor}
 We will test the application with a group of experts in the fields relevant and gather data during testing with this group and use this to evaluate how good the application is. This will serve as a good basis for analysing, triangulate and evaluate to look for relevant research results elsewhere.
\subsection{Design as a research process}
The application will be developed in two main parts, which is central to this research. The first part is development and the second is testing/evaluating the results.

Implementing the application will be developed over three separate cycles; during the first we will develop a prototype where just the basic features are present. The second cycle will implement the collaboration features in the prototype created in the first cycle and during the third we will be focusing on integrating all the features with each other and create a tool for seamless collaboration sessions. While implementing the application we will use it ourself during development

We will then do a iteration where we do testing on the application, both the case study and expert review in that order. We feel that evaluating them separately will enhance their feedback and we can compare them to see if any patterns emerge.
\subsection{Research communication}
The proof of concept application will be available for testing along with a user guide and the whole process will be well documented and our research results will be made available.
We also expect to find some limitations which will be documented along the research results.