%!TEX root = main.tex
%� Functional Requirements: Detail the requirements that you have to fulfill in order
%to complete the task.
\chapter{Functional Requirements}
\label{cha:funcreq}
% What is functional requirement?
The functional requirements below describes what the applications must implement to be considered a functional prototype.
% What is priority?
Each of the requirements are prioritised according to how important they are for the project as a whole with a three step scale.

\vspace{0.5cm}
\begin{itemize}
    \item[\textbf{H}]{High describes something that is vital for the game to function properly.}
    \item[\textbf{M}]{Medium describes something that, while not absolutely vital to the game, omission will severely limit gameplay.}
    \item[\textbf{L}]{Low describes a requirement that will enhance the game if present, but do not constitute something that a player will miss if not present.}
\end{itemize}
\vspace{0.5cm}


For a small project such as this, we feel that the three step scale gives us enough data for decisions without sacrificing too much fidelity. While the gap between M and H is fairly large, we have no need for the ability to differentiate between a 76 and 77 on a 1-100 scale.

In addition to prioritising the requirement based on importance, the requirements will also be prioritised according to how hard they are to implement. These priorities will use the same L-M-H scale as described below.

\vspace{0.5cm}
\begin{itemize}
    \item[\textbf{H}]{High is something that will take more than 10 hours to implement.}
    \item[\textbf{M}]{Medium is something that will take between 5 and 10 hours to implement.}
    \item[\textbf{L}]{Low is something that will take less than 5 hours to implement.}
\end{itemize}
\vspace{0.5cm}

What you then end up with is a way to perform a \emph{cost/benefit} analysis for each of the requirements. A requirement that is hard to implement properly, but is of little importance to the project as a whole could be dropped in benefit for a more important one (or a less difficult one), if time and/or costs involved are deemed to be to large.


% What is love?
% ♥
\begin{itemize}
    \vspace{0.5cm}
	\item[\textbf{FR1}] Change difficulty/size of ocean space. \\
        \textit{\small{The user shall be able to choose between different difficulty levels. When a level is chosen, the game creates the grid with a given size.}}

        \begin{tabular}{| l | p{7cm} |}
            \hline
            \rowcolor[gray]{0.8}
            \textbf{Impact} & \textbf{Values} \\
            \hline
            Importance priority & \textbf{M} - Not very essential, but makes the game more fun to play. \\
            Difficulty & \textbf{M} - Can be time consuming to implement. \\
            \hline
        \end{tabular}
    \vspace{0.5cm}
    \item[\textbf{FR2}] Set/change player name. \\
        \textit{\small{The player shall be able to change their name.}}

        \begin{tabular}{| l | p{7cm} |}
            \hline
            \rowcolor[gray]{0.8}
            \textbf{Impact} & \textbf{Values} \\
            \hline
            Importance priority & \textbf{L} - Not essential. \\
            Difficulty & \textbf{L} - Takes short time to implement. \\
            \hline
        \end{tabular}
    \vspace{0.5cm}
    \item[\textbf{FR3}] Game over. \\
        \textit{\small{The game ends if a player gets all of their ships destroyed. The winner is the player with any ships left.}}

        \begin{tabular}{| l | p{7cm} |}
            \hline
            \rowcolor[gray]{0.8}
            \textbf{Impact} & \textbf{Values} \\
            \hline
            Importance priority & \textbf{H} - Very essential. \\
            Difficulty & \textbf{M} - Can be time consuming. \\
            \hline
        \end{tabular}
    \vspace{0.5cm}
    \item[\textbf{FR4}] Place ones ships at the start of the game. \\
        \textit{\small{The player shall be able to place his/her ships at the start of the game and only then. During gameplay the ships must be stationary. He/she can choose to use any of the ships given at start, as seen in the table in chapter \ref{shiptable}.}}

        \begin{tabular}{| l | p{7cm} |}
            \hline
            \rowcolor[gray]{0.8}
            \textbf{Impact} & \textbf{Values} \\
            \hline
            Importance priority & \textbf{H} - Very essential. \\
            Difficulty & \textbf{H} - Will be time consuming. \\
            \hline
        \end{tabular}
    \vspace{0.5cm}
    \item[\textbf{FR5}] Play audio. \\
        \textit{\small{Play audio when a ship is hit and when the player misses.}}

        \begin{tabular}{| l | p{7cm} |}
            \hline
            \rowcolor[gray]{0.8}
            \textbf{Impact} & \textbf{Values} \\
            \hline
            Importance priority & \textbf{M} - Not essential, but will improve the game. \\
            Difficulty & \textbf{M} - Can be time consuming. \\
            \hline
        \end{tabular}
    \vspace{0.5cm}
    \item[\textbf{FR6}] Hit enemy ships. \\
        \textit{\small{A player must be able to fire on his/her turn onto one of the tiles that have not yet been fired upon.}}

        \begin{tabular}{| l | p{7cm} |}
            \hline
            \rowcolor[gray]{0.8}
            \textbf{Impact} & \textbf{Values} \\
            \hline
            Importance priority & \textbf{H} - Very essential. \\
            Difficulty & \textbf{L} - Not time consuming. \\
            \hline
        \end{tabular}
    \vspace{0.5cm}
    \item[\textbf{FR7}] Register hits on friendly ships. \\
        \textit{\small{A player shall be able to observe if the enemy player hits one of his/her ship and thus knowing the "score".}}

        \begin{tabular}{| l | p{7cm} |}
            \hline
            \rowcolor[gray]{0.8}
            \textbf{Impact} & \textbf{Values} \\
            \hline
            Importance priority & \textbf{H} - Very essential. \\
            Difficulty & \textbf{L} - Not time consuming. \\
            \hline
        \end{tabular}
\end{itemize}

