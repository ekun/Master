\section{Focus Group}
Out focus group consisted off eight people from the age of 22 to 24, they had from two to four years experience of agile-development and are all students at NTNU.

\subsection{Why they did not use it?}
The group was intended to test use the system over a period of time, but expressed that the stresslevel they where under during the test periode and the ammount of work they had remaining made them focus on that rather than testing the tool for us. They also expressed that since the tool was made availible to the so late in the process and the fact that they already had a rutine worked in and they forgot to include the tool in that rutine.

The fact that they are close to deliver the project made the testing even less prioritized, than it normally would be.

\subsection{If they would have used it}
We then showed the group how we envisioned the tool to be used and then asked the group if they had used to tool as we intended them to. Then we asked the group if they would have found the tool usefull.

When discussing tags, the group stated that they felt this might become 'off topic' and that team-members could use tags not relevant at all. This would the corrupt the tagclouds since 'off topic' tags would gain magnitude since people could use the same tags for what ever they commited, but when tagging issues f.ex

\subsection{What could be different}

\subsection{Comments}

\textbf{Om de skulle bruke det}
Bør ikke ta mye tid
Ikke mye jobb
Kjapt og oversiktlig
 * Tagcloud engasjerer
 * Burde blitt bedre når folk blir 'vant' til å tagge
 * Tror at fritt-tagging / guidelines valgt av prosjektgruppa ville fungert veldig bra.
 ** Kan bli useriøst om folk bidrar med uviktige tags ** 

 Team vs Me <-- Tags bør plasseres ganske likt / fargene bør være like
				Vanskelig å se hvor lignende ord osv er
				Ser nytteverdiene, men med maglene over ville en hatt en possitiv effekt

Dailyreflection kan være meget nyttig. <-- De glemte hva de hadde gjort uka i forveien.
	* Kunne gjerne vært mer klarrerende hva spørsmålene gikk på


 * Mood / tema inn i commit tags
 Mood osv burde vært dratt inn

 ** Workshops **
 Vil gjøre retroskeptivene mer styrt og hjelper med å dra diskusjonen i gang, setter også opp mer struktur.
 Gjør at en bruker mer tid og er mer nøyaktig i stressa situasjoner.

 Ville tro at det var et veldig godt verktøy til retrospektiver, men siden de ikke har testet dette er det ingen måten å få dette verifiserres.
 
 Qoute: 'Virker som et fornuftig og godt verktøy' H

 Qoute: 'Verktøyet virker godt for prosjektleder som ikke sitter handsdown i koden' F

** Workshop Preperation **
 MEGET nyttig før retrospektiv.

\textbf{Why they didn't use it}
* Fikk det introdusert veldig seint
	* Har hatt mye å gjøre
	* Har glemt det
* Fikk ikke jobbet det inn i rutinen.
* Har prioritert andre ting.


* Did not use Github for milestones and issues, used trello instead.
	* Did used it for documentation.

\textbf{Kommentarer}
Enkle kommentarer om verktøyet.

\textbf{Workshop}
Lurer på om de 5 mandatory spørsmåla skal legges inn i daily reflection for å svare på disse da.
	* Vi mener vel at det går vil gjøre den daglige litt for 'tung' å gjennomføre.


\textbf{GIT}
* Masse unyttig bs om git(NOK EN GANG)
* < 50 commit message length, unsure about the benefits of this
* Impact er ikke nyttig egentlig
	* Kunne begrenset til et sett med filer.
	** Ammend på commits for 2 uker siden, forbanna crazy! Herre så idiotisk.......
		* ey, systemet vårt plukker opp.
* Fixup 'sletter' de gamle, mens squash og ammend funker?!

* Det argumenteres av enkelte at git gir det sammen, mens de andre påpeker at det her gir er mye mer og fullstendig oversikt av hva som har blitt jobbet med.



******* Masse piss om branches ********

** Issue **
Vil gjerne ha ting i historisk rekkefølge som på github, ser ingen gevinst i måten vi har sortert det på.

HELVETE som folk prater i munnen på hverandre.
--> Github er awesome --> STORE REPOS ER KONGE

** Team **
Bør muligens kunne legges til flere repos til et team.


\textbf{Future work}
* Dailyreflection *
	Burde kunne skrive inn hele dagen så "låses" noten når dagen har gått over. (BRA)
	Ville gjerne ha valget å motta på mail
	Oversikt av commits / teamet hashtags
	Muligheten for å kalle dagen isteden for dato. (MEGET BRA)
	Klokkeslett og ID er forbanna idiotisk

	* Notification *
		En mer invasive melding på hver side isteden for notification på side.
			Mer irriterende rett og slett

* Daily note review, kan vise tagclouden fra den aktuelle dagen. (Deg vs team)
	

* Moodgraph med mange personer vil muligens være unyttig


* [Tema for commit] f.eks GUI, BACKEND/GitSyncer.java


* Før hver sprint planlegge temaer / tags slik at det blir mer 'scrum-vennlig' og kan brukes mer prosjektstyrende

*** Viktig å sette mood i sammenheng med hva folk har jobbet med. *** 